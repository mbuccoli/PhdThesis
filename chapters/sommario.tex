% !TeX spellcheck = it_IT
\chapter*{Sommario}
\lettrine{I}{n} questi anni, l'evoluzione della tecnologia e delle reti di connessione hanno portato ad una rivoluzione negli scenari di fruizione della musica. Nel giro di un decennio, milioni di brani musicali sono diventati disponibili attraverso molteplici piattaforme di streaming o diffusione, rendendo necessari nuovi strumenti di navigazione, suggerimenti automatici su cosa ascoltare, negozi virtuali, etc. Il \textit{Music Information Retrieval} (MIR) \`e un campo di ricerca multidisciplinare che affronta le problematiche relative all'ideazione di nuove strategie di ricerca, navigazione e annotazione al fine di implementare gli strumenti necessari al supporto di questi scenari, come ad esempio gli algoritmi di annotazione e classificazione automatica dei brani.

La progettazione degli strumenti per questi scenari del MIR richiede di poter descrivere la musica da due diversi punti di vista: quello legato alla semantica e quello legato al segnale. Il primo riguarda il modo in cui percepiamo soggettivamente e interpretiamo le caratteristiche musicali, quali termini scegliamo per descriverle e come usiamo questi termini quando parliamo di musica. Il secondo invece riguarda le propriet\`a oggettive dei segnali musicali, da quelle direttamente calcolabili a quelle meno dirette, come le propriet\`a ritmiche o tonali. Questi due livelli di astrazione per\`o sono come mondi separati, che la ricerca sul MIR cerca di avvicinare e collegare.

Gli approcci moderni per lo sviluppo di applicazioni MIR devono prendere in considerazione tutti i livelli di astrazione della descrizione musicale, per poter sviluppare le applicazioni MIR. Ci\`o significa che, oltre a considerare il \textit{dominio del segnale} e il \textit{dominio semantico}, questi approcci si concentrano anche sulla \textit{funzione di collegamento}. In questa tesi, ci proponiamo di seguire uno schema che prevede la formalizzazione dei dom\'in\^i del segnale e semantico e la progettazione della relativa funzione di collegamento.

La tesi muove da una discussione delle diverse modalit\`a di formalizzazione del dominio del segnale, partendo dall'estrazione di una rappresentazione del contenuto musicale basata su \textit{feature}. Questa formalizzazione richiede una profonda conoscenza delle propriet\`a musicali e delle feature in grado di catturarle. Come parte di questa formalizzazione, perci\`o, discutiamo di una nostra ricerca volta a stimare come alcune propriet\`a musicali influenzino la qualit\`a dell'esperienza di performance remota, tramite un'analisi delle feature audio. Inoltre, discutiamo di come le tecniche di \textit{apprendimento approfondito} possono essere utilizzate per estrarre (o \textit{imparare}) automaticamente un'efficace rappresentazione del contenuto musicale.

Descriviamo anche le modalit\`a di progettazione della funzione di collegamento iniziando dalle tecniche procedurali, che seguono un algoritmo progettato manualmente. Sovente per\`o, il collegamento tra i due dom\'in\^i non \`e chiara ed \`e quindi difficile riuscire a trovare una soluzione procedurale. In questi casi, \`e possibile usare tecniche di apprendimento automatico della relazione tra i due dom\'in\^i. 

Il dominio semantico pu\`o invece essere formalizzato seguendo due approcci principali: quello per categorie, che definisce quali descrittori possono rappresentare un dato brano; e quello dimensionale, che specifica anche quanto i descrittori sono in grado di rappresentare il brano. Il set di descrittori semantici pu\`o essere anche arricchito includendo la similarit\`a semantica --sia definita manualmente che appresa automaticamente-- tra i descrittori; l'insieme di descrittori e similarit\`a viene chiamato \textit{modello semantico}. Riguardo ci\`o, discutiamo anche una nostra ricerca volta ad arricchire un \textit{dataset} generico di descrittori dimensionali con informazioni, specifiche per la musica, estratte automaticamente da annotazioni degli utenti. Consideriamo anche la \textit{struttura} dei brani come un caso particolare di formalizzazione del dominio semantico.

Una volta che i componenti principali sono definiti e formalizzati, utilizziamo questo schema in una serie di scenari applicativi di complessit\`a crescente.

Il primo scenario applicativo riguarda l'analisi ed estrazione della struttura musicale. Facciamo affidamento su tecniche di apprendimento approfondito per estrarre una rappresentazione del dominio del segnale. Cos\`i facendo, riusciamo a risolvere i problemi di incertezza legati alla scelta delle propriet\`a musicali utili a descrivere la struttura di brano. Poich\'e il dominio semantico \`e invece ben formalizzato, possiamo utilizzare delle tecniche procedurali comuni nella letteratura per estrarre la struttura.

Nel secondo scenario applicativo affrontiamo il riconoscimento di bootleg, ovvero registrazioni non autorizzate di performance live. Il dominio semantico di questo scenario \`e formalizzabile assegnando ogni brano a una categoria tra le seguenti: bootleg, live ufficiale o registrazione in studio. Utilizziamo tecniche di apprendimento approfondito per formalizzare il dominio del segnale e tecniche di appendimento automatico per facilitare la progettazione della funzione di collegamento.

Il terzo scenario applicativo concerne l'annotazione automatica delle qualit\`a timbriche di violini, partendo da alcune registrazioni audio. Le qualit\`a timbriche sono descritte da una semantica ambigua, che formalizziamo tramite sei descrittori dimensionali. L'approccio dimensionale ci permette di aumentare l'espressivit\`a del modello semantico altrimenti limitata dal numero ridotto di descrittori. Come nei precedenti scenari, il dominio del segnale \`e formalizzato con tecniche di apprendimento approfondito, e legato al dominio semantico tramite tecniche di apprendimento automatico.

Nell'ultimo scenario ci occupiamo della definizione di un modello semantico che consideri le ambiguit\`a frequenti nel linguaggio naturale. Investighiamo una formalizzazione del dominio semantico in grado di modellare la \textit{polisemia}, ovvero i termini che assumono diversi significati quando usati in diversi \textit{contesti semantici}. Usiamo il nostro modello semantico, basato su contesti semantici sovrapponibili e similarit\`a semantiche diverse per contesti diversi, per implementare un prototipo di un motore di ricerca musicale: un'applicazione innovativa in grado di elaborare \textit{query} in linguaggio naturale per trovare i brani musicali desiderati.


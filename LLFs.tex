% !TeX spellcheck = en_US
\chapter{Feature representation of the signal domain}
\label{Chap:LLFs}
In order to develop systems that allow users access to the music libraries, it is crucial to define and formalize the problem from the perspective of the content, i.e., the signal domain. The signal domain involves three main levels of information: the physical level, which is related to the origin and propagation of the audio wave and can be analyzed with signal processing techniques; the perceptual level, which concerns how our body perceives and processes the audio information; and the musical level that describes musicological aspects. In this Chapter we  present the state of the art and the theoretical background needed to formalize the audio signal domain.

The MIR community developed and designed several techniques to describe the signal domain by automatically extracting features from the audio signal. Such features describe different aspects of the sound at various levels of abstraction, from those related to the energy, the spectrum or the timbre, to those regarding musical aspects of the music performance. A description of all the features employed in Music Information Retrieval is beyond the scope of this work, and in the following Sections we discuss the most popular features and those that we use throughout the thesis.

In Section \ref{sec:LLFs:hand-crafted}, we describe the so-called \textit{hand-crafted} and \textit{model-based} features, that are manually designed by scientists by means of exact mathematical formulations or of perceptual and musicological models, respectively. These features are usually divided into low-level features (LLFs), that capture a specific characteristics of the audio signal by means of the exact mathematical formulations, and mid-level features (MLFs) that make use of the musicological aspects of the songs. For the sake of brevity, in the following discussion we will refer to both kind of features as hand-crafted features, since the perceptual or musicological models of the model-based features have been manually designed as well.

%Such features carry a low level of semantics, since they can only describe basic properties of the signal. Nevertheless, their design and interpretation can be used for real world applications. 
In order to better explore the potential of such features, in Section \ref{sec:NMP} we present a research activity on Networked Music Performance, where the design of a network architecture benefits from a feature-based analysis of the (music) signal domain of the problem.
Finally, in Section \ref{sec:LLFs:learned}, we provide a theoretical background for the \textit{learned} features, that are automatically extracted by means of deep learning techniques, and we focus on the deep architecture we used for our work. %We also present a brief overview of deep learning techniques from the state of the art that have been used to extract a representation of the music signals.


\section{Hand-Crafted and Model-Based Features}\label{sec:LLFs:hand-crafted}
When analyzing a problem, it is important to understand which aspects are more useful to characterize the domain of the problem and which features are most suitable to capture them. For this reason, it is crucial to understand the insight we can infer from the different features. The hand-crafted and model-based features provide a formalization of the problem related to the musical content, by providing a reliable description of the audio signal and of the underlying musical aspects. Such low-level representation is a valuable resource for   %and are designed to be interpretable by 
researchers, scientists and musicians, who can interpreter the carried information.  

Due to the strong relevance of hand-crafted in Music Information Retrieval, several tools have been developed to automatically extract them, such as the MIRToolbox for Matlab \cite{Lartillot2007}, LibRosa for Python  \cite{brian_mcfee_2015_18369}, Marsyas \cite{tzanetakis2000marsyas} and Essentia\cite{bogdanov2013essentia} for C++ and Sonic Annotators \cite{chris2010a} as a command line routine. 

%The hand-crafted features aim at extracting rather objective features, that are supposed to assume the same values for the same songs. Nevertheless, their values can change depending on the pre-processing performed to the music piece, including using different audio quality values. In \cite{urbano2014effect} it is shown how different audio qualities affect the range of the values of the extracted features and therefore their reliability. For this reason, when composing a dataset, it is important to hold the same settings for sampling frequency, sample bitrate, and a regularization is usually performed over the set, as better discussed in Section \ref{sec:ML:regularize}. In light of these requirements, it is necessary to dedicate a careful and significant effort in the collection of songs in a dataset for music analysis or training. The \textit{learned} features attenuate this issue by attempting to extract rather abstract and subjective characteristics.

In the following, we will describe and visualize some of the features used by the MIR community. We select two highly different music excerpts for the visualization, in order to compare the outcome of the feature extraction. The first song is \textit{Orinoco Flow} by the artist Henya, which can be described as a slow, flowing, calm and harmonic song; the second song is \textit{Down with the Sickness} by the band Disturbed, that has an aggressive mood and it is fast, noisy and stuttering \cite{Buccoli2013}. The considered excerpts are included in the dataset presented in \cite{kim2008moodswings}.

\subsection{Low-Level Features}\label{sec:LLFs:LLFs}
The Low-Level Features provide some insight on the audio signal that needs interpretation for a manual analysis, due to the low level of abstraction from the signal. %They are absolutely objective, since they only depend on the signal and not on a (possibly biased) model.
% % STATO DELL'ARTE
LLFs have been widely employed in Music Information Retrieval for several tasks. %, since they are easy to extract and reliable.  
In \cite{mckinney2003features,Kim2005} the authors present an overview of the use of LLFs in MIR. In \cite{eck2008automatic}, the authors use a set of spectral features for the generic automatic annotation of songs with semantic descriptors, while in \cite{schmidt2010prediction} the LLFs are used to predict the distribution of the mood content of a song. The description provided by LLFs have also been used to compare two audio recordings by defining a similarity function between their feature-based representations \cite{pampalk2005improvements}. This approach is used in \cite{Kim2004} to perform audio classification with a query-by-example: an audio recording is presented to a system, which retrieves the acoustically-similar recordings in a dataset. Low-Level Features are also widely employed for the analysis of the music structure as a frame-level description of a song \cite{levy2008structural,ong2005semantic,kaiser2012music} or for music classification \cite{Bestagini2013}.

% % DESCRIZIONE FEATURES
In this work, we mainly focus on spectral features, which are extracted from the frequency-domain representation of the audio signal and can provide some insight on the timbral aspects of the sound. We compute the frequency-domain representation of the audio signal by means of the Short-Time Fourier Transform (STFT), which is the Fourier Transform computed on possibly overlapping and windowed frame of the time-domain signal. In this Chapter, we will refer to the STFT of a generic audio signal $s(n)$ as $S$, with $S_l$ the generic $l$-th frame, $S(k)$ the STFT at the $k$-th frequency bin, and $S_l(k)$ the $k$-th frequency bin component at the $l$-th frame. Finally, we will refer to the magnitude of the spectrum as $|S|$.% and to the angle as $\angle S$.

The more basic features to capture are the four statistics moments of the spectrum, namely the \textit{Spectral Centroid}, \textit{Spectral Spread}, \textit{Spectral Skewness} and \textit{Spectral Kurtosis}. These moments are, in fact, widely used Spectrum descriptors \cite{Kim2005,Zanoni2014,Zanoni2012}.

\begin{figure}[t]
        \centering       
          \subfloat[Spectral Centroid for \textit{Down with the Sickness}]{\includegraphics[width=.45\textwidth]{img/LLFs/Down_centroid}\label{fig:LLFs:down_centroid}} \hfil
          \subfloat[Spectral Centroid for \textit{Orinoco Flow}]{\includegraphics[width=.45\textwidth]{img/LLFs/Henya_centroid}\label{fig:LLFs:henya_centroid}}
          \caption{Spectral Centroid for the two songs.}
          \label{fig:LLFs:centroid}          
\end{figure}

The \textit{Spectral Centroid} ($SC$) represents the ``center of gravity'' (first moment) of the magnitude spectrum, i.e., the frequency at which the energy of the spectrum for lower frequencies and for higher frequency is about the same \cite{Li2000}. The $SC$ is computed as:
\begin{equation}\label{eq:FSC}
SC_l = \frac{\sum\limits_{k=1}^{K}f(k)|S_l(k)|}{\sum\limits_{k=1}^{K}|S_l(k)|} \; ,
\end{equation}
where $K$ represents the total number of frequency bins, $f(k)=F_s \cdot k/K $ is the frequency corresponding to the $k$-th bin and $F_s$ is the sampling frequency of $s(n)$. The Spectral Centroid gives us an insight on where the frequency components of the magnitude spectrum are more distributed and hence which frequencies are likely to be perceived as predominant. The lower is the Spectrum Centroid, the darker will be the sound and the higher is the Spectrum Centroid, the brighter will be the sound. For this reason, the Spectrum Centroid can be seen as a descriptor of the brightness of the sound. Spectral Centroids for two reference songs are shown in figure \ref{fig:LLFs:centroid}. We notice that the average Spectral Centroid of \textit{Orinoco Flow} (Fig. \ref{fig:LLFs:henya_centroid}) is lower than in the \textit{Down with the Sickness} (Fig. \ref{fig:LLFs:down_centroid}), that might seems counter-intuitive. However, it is worth remembering that we are considering a \textit{bright} or \textit{dark} sound from a timbral perspective, which should not be confused with the emotional perspective. We present a work that address such kind of ambiguities in Chapter \ref{Chap:DCSM}. In this case, the Centroid in  \textit{Orinoco Flow} is around 2000 Hz, which can be due to various low-pitch harmonic instruments that play in the song. The high Centroid for \textit{Down with the Sickness}, instead, can be caused by high-pitch percussive sounds or electric guitars.

\begin{figure}[tb]
        \centering
      \subfloat[Spectral Spread for \textit{Down with the Sickness}]{\includegraphics[width=.45\textwidth]{img/LLFs/Down_spread}\label{fig:LLFs:down_spread}} \hfil
      \subfloat[Spectral Spread for \textit{Orinoco Flow}]{\includegraphics[width=.45\textwidth]{img/LLFs/Henya_spread}\label{fig:LLFs:henya_spread}}
      \caption{Spectral Spread for the two songs.}
      \label{fig:LLFs:spread}          
\end{figure}

The second statistic moment of the distribution of the spectrum is the \textit{Spectral Spread} ($SSp$) and it measures the standard deviation of the spectrum from its frequency mean, i.e., from the Spectral Centroid. The Spectral Spread is computed as :
\begin{equation}\label{eq:FSS}
SSp_l = \sqrt{\frac{\sum\limits_{k=1}^{K}(f(k)-SC_l)^2 |S_l(k)|}{\sum\limits_{k=1}^{K}|S_l(k)|}} \;.
\end{equation}
The Spectral Spread captures how much the power of the spectrum is spread over the frequencies around the Spectral Centroid. The lower is the $SSp$, the more the spectrum is distributed around its centroid and hence, the more it resembles a pure tone. On the contrary, a noisy sound is characterized by a spreader spectrum and therefore by high values of the Spectral Spread. The Spectral Spread can therefore be used as a measure of the noisiness of the audio signal. In Figure \ref{fig:LLFs:spread} we compare the Spread of the two songs. The $Ssp$ values of \textit{Down with the Sickness} (Fig. \ref{fig:LLFs:down_spread}) are quite high, so we might assume that the distribution of the spectrum has a great variance, hence possibly sounding rather noisy. On the other side, from the lower values of \textit{Orinoco Flow} (Fig. \ref{fig:LLFs:henya_spread}), combined with the lower centroid seen in Figure \ref{fig:LLFs:henya_centroid}, we might assume that the song contains low-pitch and rather harmonic sounds.
	

\begin{figure}[tb]
        \centering
          \subfloat[Spectral Skewness for \textit{Down with the Sickness}]{\includegraphics[width=.45\textwidth]{img/LLFs/Down_skewness}\label{fig:LLFs:down_skewness}} \hfil
          \subfloat[Spectral Skewness for \textit{Orinoco Flow}]{\includegraphics[width=.45\textwidth]{img/LLFs/Henya_skewness}\label{fig:LLFs:henya_skewness}}
          \caption{Spectral Skewness for the two songs.}
          \label{fig:LLFs:skewness}          
\end{figure}
	
The third statistic moment is called \textit{Spectral Skewness} ($SSk$) and it computes the coefficients of the skewness of the frequency distribution, i.e., the degree of its symmetry around the Spectral Centroid \cite{Tanghe2005}:
\begin{equation}\label{eq:FSSK}
SSk_l = \frac{\sum\limits_{k=1}^{K}(f(k)-SC_l)^3 |S_l(k)|}{K SS_l^3} \; .
\end{equation}
An exactly symmetric frequency distribution results in a zero Spectral Skewness. Positive values of $SSk$ indicate a distribution of the spectrum energy with a longer or fatter tail toward the higher frequencies (with respect to the Spectral Centroid). Negative values of $SSk$ indicate a longer or fatter tail of the spectrum energy towards the lower frequencies and the. Together with the Centroid and the Spread, Spectral Skewness gives us an insight on the frequency distribution, but it is hard to derive a general interpretation from it. However, from the visualization in Figure \ref{fig:LLFs:skewness}, it is clear that the distribution of the Orinoco Flow's spectrum presents a tail toward the higher frequencies, (Fig. \ref{fig:LLFs:henya_skewness}), while \textit{Down with the Sickness} presents lower values and sometimes negative (Fig. \ref{fig:LLFs:down_skewness}), that indicates the distribution of the spectrum is more uniform than those from Oricono Flow.

\begin{figure}[tb]
        \centering
        \subfloat[Spectral Kurtosis for \textit{Down with the Sickness}]{\includegraphics[width=.45\textwidth]{img/LLFs/Down_kurtosis}\label{fig:LLFs:down_kurtosis}} \hfil
      \subfloat[Spectral Kurtosis for \textit{Orinoco Flow}]{\includegraphics[width=.45\textwidth]{img/LLFs/Henya_kurtosis}\label{fig:LLFs:henya_kurtosis}}
      \caption{Spectral Kurtosis for the two songs.}
      \label{fig:LLFs:kurtosis}          
\end{figure}

The fourth statistics moment is called \textit{Spectral Kurtosis} $SK$ and it indicates the size of the tails of the frequency distribution:
\begin{equation}\label{eq:FSK}
SK_l = \frac{\sum\limits_{k=1}^{K} (|S_l(k)|-SP_l)^4 }{K \cdot SS_l^4} -3 \; .
\end{equation}
A normal frequency distribution results in zero Spectral Kurtosis, thanks to the $-3$ offset. Positive values of $SK$ indicates relatively large tails in the distributions, while distributions with small tails exhibit negative values. The Spectral Kurtosis can be interpreted as an indication of the deviation from the normal distribution. It is clear from Figure \ref{fig:LLFs:kurtosis} that the distribution of the spectrum of \textit{Orinoco Flow} has rather large tails (Fig. \ref{fig:LLFs:henya_kurtosis}), while \textit{Down with the Sickness} closely resembles to a normal distribution, i.e., a noisy sound (Fig. \ref{fig:LLFs:down_kurtosis}).


Two more spectral features are the \textit{Spectral Entropy} ($SE$) \cite{Lartillot2007}, and \textit{Spectral Flatness} ($SF$) \cite{Kim2005}, both providing an indicator of the noisiness of the sound.

\begin{figure}[tb]
        \centering
   \subfloat[Spectral Entropy for \textit{Down with the Sickness}]{\includegraphics[width=.45\textwidth]{img/LLFs/Down_entropy}\label{fig:LLFs:down_entropy}} \hfil
 \subfloat[Spectral Entropy for \textit{Orinoco Flow}]{\includegraphics[width=.45\textwidth]{img/LLFs/Henya_entropy}\label{fig:LLFs:henya_entropy}}
 \caption{Spectral Entropy for the two songs.}
 \label{fig:LLFs:entropy}          
\end{figure}

The \textit{Spectral Entropy} ($SE$) applies the Shannon's entropy definition \cite{shannon2001}, commonly used in information theory context, to estimate the flatness of the spectrum. The Spectral Entropy is commonly computed \cite{MIRToolbox} with a factor of normalization that provides an estimation which is independent on the length of the frame:
\begin{equation}
	SE_l = -\frac{\sum\limits_{k=1}^{K}|S_l(k)|\log |S_l(k)|}{\log K}.
\end{equation}
The spectral entropy captures the uncertainty of the distribution of the spectrum and hence its flatness. The maximum entropy is indeed obtained with a perfect flat spectrum, while the minimum entropy is given by a very sharp peak in the spectrum with a low background noise. This is the reason why the Spectral Entropy is used as an indicator of the noisiness of the sound.

\begin{figure}[tb]
        \centering
       \subfloat[Spectral Flatness for \textit{Down with the Sickness}]{\includegraphics[width=.45\textwidth]{img/LLFs/Down_flatness}\label{fig:LLFs:down_flatness}} \hfil
       \subfloat[Spectral Flatness for \textit{Orinoco Flow}]{\includegraphics[width=.45\textwidth]{img/LLFs/Henya_flatness}\label{fig:LLFs:henya_flatness}}
       \caption{Spectral Flatness for the two songs.}
       \label{fig:LLFs:flatness}          
\end{figure}

The \textit{Spectral Flatness} ($SF$) is also used as an estimate of the noisiness of the sound \cite{Rottondi2015}, as an indicator of the degree of flatness of the spectrum. It is computed as the ratio between the geometric and the arithmetic mean
\begin{equation}
	F_{SF_l} = \frac{\sqrt[K]{\prod\limits_{k=0}^{K-1}|S_l(k)|}}{\sum\limits_{k=1}^{K}|S_l(k)|},
\end{equation}
which provides an estimation of the similarity between the magnitude spectrum of the signal frame and the flat shape inside a predefined frequency band. 


A visualization of Spectral Entropy and Spectral Flatness is shown in Figures \ref{fig:LLFs:entropy} and \ref{fig:LLFs:flatness} respectively. Both features capture higher values of noisiness for \textit{Down with the Sickness} than for Orinoco Flow, which is confirmed by the listening of the two pieces.

Spectral features are important %to provide an insight on the sound of the signal. They are designed 
to capture properties of the spectrum from which we can infer the perceptual qualities of the sound. % that are perceptually meaningful, hence they measure some perceptual values from the signal. 
We can however reverse the approach by designing features that measure and return a signal from the perceptual representation of the sound. We can do so by exploiting the information on how the human auditory system works. From the studies on psychoacoustics, we know that the human perception of frequency is logarithmic, as well as the perception of loudness. In particular, the decomposition of a sound in the correspondant frequency components is performed within the spiral-shaped cavity named \textit{cochlea} in the inner ear. The cochlea acts as a filter bank for different ranges of frequencies \cite{miotto2011}. 

The \textit{Mel-Frequency Cepstral Coefficients} (MFCCs) are a set of descriptors that exploit a model from psychoacoustics on the human auditory system to provide perceptual features. %In order to take into account the information on the auditory system, the \textit{Mel-Frequency Cepstral Coefficients} (MFCCs) have been designed. MFCCs rely on a model of the auditory system to provide perceptually meaningful features. 
The MFCCs have become one of the most widely used descriptors for Music Information Retrieval \cite{muller2007information, muller2015fundamentals}.  Given a generic frame of a STFT $|S_l|$, the MFCCs are computed by performing the following processing. In the following, we will neglect the notation of the $l$-th frame for the sake of clarity. First, a mel-filter bank, i.e., a filter bank whose bands model the auditory response of the cochlea, is applied to the frame. The mel scale follows the logarithmic perception of the frequencies, so there are more filter banks in the lower frequencies than in the higher ones. The log-\textit{Power Spectrum} $E_k$ is computed for each of the band of the mel-filter bank, where the logarithmic scale is used in order to take the human perception of loudness into consideration. Finally, the Discrete Cosine Transform (DCT) is applied to the Power Spectrum in order to extract a complete yet compact representation of it:
\begin{equation}
%\begin{array}{rcl} 
c_i = \sum_{k=1}^{N_c}  \log(E_k) \cos \left[i \left(k-\frac{1}{2}\right) \frac{\pi}{N_c} \right]  \;\text{ with }\; 1\leq i\leq K_c, 
%\end{array}
\end{equation}
where $c_i$ is the $i-th$ MFCC component, $E_k$ is the spectral energy measured in the critical band of the $i-th$ mel-filter, $N_c$ is the number of mel-filters and $K_c$ is the amount of cepstral coefficients extracted from each frame, which is typically 13. Back to the indication of the frames, the vector $\mathbf{c_l}$, is composed of the $K_c$ cepstral coefficients computed for the $l-th$ frame, is a valid descriptor of the timbre of the underlying sound. The MFCCs extracted from the two songs of examples are %are not easily directly interpretable, as seen from the representation for the two songs into analysis are 
shown in Figure \ref{fig:LLFs:mfcc}. 


The Low-Level Features are computed over a frame representation of the audio signal, with typical values of the frame length are 1024, 2048 samples, i.e., 50-100 ms for a 44,1 kHz sampling frequency. This leads to a tremendous amount of data for the music representation, which is often not manageable because of memory and computational issues. A typical approach to create a more compact representation is to \textit{pool} together several frames to compose a larger analysis window. This approach makes the features less accurate and precise in the time domain, but it also increases the quality of the captured features by smoothing possible outliers. The traditional pooling techniques are based on the average, the median or the maximum values over frames within fixed-length analysis windows. Other pooling techniques benefit from the knowledge of the musicological aspects of the content, such as the rhythm, which are discussed in the following Section.

%The pooling of the frames can also be used to embed other information from the signal, such as the musicological aspects. In the foll Another pooling approach concerns to use the information from music rhythm to create a feature representation that also embeds some musical aspects. In order to do so, we need to extract some features from the signal domain that provide a representation of such musical aspects, by means of the Mid-Level Features described in the following Section.

\begin{figure}[tb]
        \centering
      \subfloat[MFCCs for \textit{Down with the Sickness}]{\includegraphics[width=.45\textwidth]{img/LLFs/Down_mfcc}\label{fig:LLFs:down_mfcc}} \hfil
      \subfloat[MFCCs for \textit{Orinoco Flow}]{\includegraphics[width=.45\textwidth]{img/LLFs/Henya_mfcc}\label{fig:LLFs:henya_mfcc}}
      \caption{MFCCs for the two songs.}
      \label{fig:LLFs:mfcc}          
\end{figure}


\subsection{Mid-Level Features}\label{sec:LLFs:MLFs}
The Mid-Level Features (MLFs) provide a higher level of abstraction from the audio signal by including some musical knowledge into the feature extraction process. Various approaches have been proposed to extract musically meaningful features, using both model-based approaches and machine learning techniques. Several of the issues related to the Mid-Level Features are still open and the MIREX competition holds every year to compare the performance of approaches proposed by several research groups within the MIR community \cite{downie2014ten}.

The MLFs capture several aspects of the music. Two of the main aspects that can be described as MLFs are the harmony and the rhythm. The former is composed of the pitch of notes, chords, keys and the modes, the latter extracts beats and bars, and the rhythmic patterns from the music pieces \cite{muller2007information,muller2015fundamentals,bello2010identifying,Nieto2D}. The MLFs can be %used as a representation of the signal domain or as a
also a helpful resource for the design of other algorithms. %Moreover, some MLFs are integrated during the processing of other algorithms, providing useful information on the music information. 
As an example, tracking the beats can be useful with regards to the pooling approaches described in the previous Section \ref{sec:LLFs:LLFs}, where a common approach is to pool together the frames that occur between two consecutive beat. This \textit{beat-synchronized} representation is independent on the variation of tempo and hence has been proved to be particularly useful for cover identification and structure analysis tasks \cite{Ellis2007,Nieto2D}.

The MLFs are either directly extracted from the audio signal or estimated from an intermediate symbolic representation of the songs where the components from the sheet music %basic musical aspects 
are structured and explicit. %While the audio content is easier to retrieve than the 
There is a lower availability of symbolic representation of music pieces, since it requires a manual effort to be exactly annotated. Nevertheless, when available, such representation %symbolic representation, the latter 
is more reliable for the analysis of complex or abstract information, such as the harmonic or rhythmic complexity, the degree of surprise or the difficulty in listening. For this reason, some tools for the automatic extraction of features from symbolic notation have been developed, such as the MIDI Toolbox \cite{Eerola2004}.


\begin{figure}[tb]
	\centering
	\subfloat[Chromagram for \textit{Down with the Sickness}]{\includegraphics[width=.45\textwidth]{img/LLFs/Down_chroma}\label{fig:LLFs:down_chroma}} \hfil
	\subfloat[Chromagram for \textit{Orinoco Flow}]{\includegraphics[width=.45\textwidth]{img/LLFs/Henya_chroma}\label{fig:LLFs:henya_chroma}}
	\caption{Chromagram for the two songs.}
	\label{fig:LLFs:chroma}          
\end{figure}


As far as the harmonic features are regarded, the most widely used MLFs are the \textit{chroma} features, also called \textit{Pitch Class Profile} (PCP), which represent the distribution of pitch classes in a certain frame of analysis, and the \textit{chromagram}, which collect the chroma features through the frames of a song. In the equel tempered scale, which is the main scale used in western popular music, two notes belong to the same pitch class when the ratio of their fundamental frequencies is an integer power of two (e.g., $f_1=f$, $f_2=2f$). The equal tempered scale consists of twelve pitch classes, each of which is identified with a name, from C to B through the semi-tones. When shifting a melody by an octave (i.e., doubling or halving the frequency of the notes), in fact, the melody keeps the same harmony and key of the original. It is reasonable therefore to summarize the harmonic content of a song by considering the pitch classes of the notes, regardless of the original octave.


The chroma features are built by first applying a filter bank to the log-magnitude spectrum of a frame, with the filters' bands centered in the fundamental frequencies of the notes from the equal tempered scale. The results of the pass-band filters are then collapsed in a 12-bin histogram-like structure where each bin represents a pitch class. The chroma features and the collected chromagram are often the first step for further processing regarding chord or harmony recognition or music structure analysis \cite{Nieto2D, Digiorgi2013}. A visual representation is shown in Figure \ref{fig:LLFs:chroma}. The presence of harmonic sounds in \textit{Orinoco Flow} allows us to clearly see the predominant notes in the chromagram (Fig \ref{fig:LLFs:henya_chroma}), while the chromagram extracted from \textit{Down with the Sickness} appears more confused (Fig \ref{fig:LLFs:down_chroma}).

\begin{figure}[tb]
	\centering
	\subfloat[Tempo for \textit{Down with the Sickness}]{\includegraphics[width=.45\textwidth]{img/LLFs/Down_tempo}\label{fig:LLFs:down_tempo}} \hfil
	\subfloat[Tempo for \textit{Orinoco Flow}]{\includegraphics[width=.45\textwidth]{img/LLFs/Henya_tempo}\label{fig:LLFs:henya_tempo}}
	\caption{Tempo for the two songs.}
	\label{fig:LLFs:tempo}          
\end{figure}

One of the most widely used rhythmic features is the \textit{tempo}, which represents the speed of execution of a given piece. As a descriptor, tempo is correlated with other semantic features \cite{Buccoli2013}, e.g., related to mood or dynamics: a piece with a slow tempo is more likely to be described as \textit{calm}, or \textit{romantic}, while a faster tempo can be correlated with more active moods. Tempo is specified in beats per minute (BPM), i.e., how many beats must be played in a minute, where the beats are defined as  \textit{the temporal unit of a composition, as indicated by the (real or imaginary) up and down movements of a conductor's hand}\cite{harvardDictionary}. We can estimate the instantaneous tempo computing the time interval between two consecutive beats, or we can estimate the tempo over longer time windows. For example, in the Figure \ref{fig:LLFs:tempo} we can see an analysis of tempo over 3 second-windows with 50\% overlap, which confirms that \textit{Orinoco Flow} is slower than \textit{Down with the Sickness} (Figures \ref{fig:LLFs:down_tempo}, \ref{fig:LLFs:henya_tempo}), but it also highlights the difficulty of the task of automatic mid-level feature modeling and extraction, since the estimated tempo is quite unstable, while the real tempo of the excerpts is not. 


In this work we also use two rhythmic features as indicator of the rhythmic complexity of a music piece. Among the various proposed indicators \cite{shmulevich2000}, we used the \textit{Event Density} and the \textit{Rhythmic Complexity}. %These features are not commonly used in the literature, since the MIR community has proposed several features for the rhythmic complexity metric \cite{shmulevich2000}, while no common consensus has been reached.
Such indicators both analyze the rhythm by considering only the \textit{onsets} of the notes. %, i.e., the pure temporal event when they occur, regardless of their pitch.
%In order to discuss the formalization of these features, it is important to define the basic entity of the rhythm, i.e., the onset. An onset is defined as a pure temporal note event, regardless of its pitch, and it is identified by its duration and the moment when it occurs.

The \textit{Event Density} ($ED$) is defined in \cite{Lartillot2007} as the average number of onsets per second:
\begin{equation}
ED = \frac{NO}{T},
\label{eq:LLFs:ED}
\end{equation}
where $NO$ is the number of onsets and $T$ is the duration of the musical piece. In order to 
make the $ED$ values for polyphonic parts comparable with the ones computed over monophonic parts or with purely percussive parts, when several notes are played in the same instant, e.g., in a chord, they are counted as one onset.  

The \textit{Rhytmic Complexity} ($RC$) \cite{povel} estimates the complexity of the rhythmic pattern of a part as a weighted sum of different factors that are believed to contribute to make a part rhytmically complex. The Rhytmic Complexity is computed as: 
\begin{equation}\label{eq:RC}
RC= w_1 \; H + w_2 \; ED + w_3 \; \sigma + w_4 \; A,
\end{equation}
where $w_1, \ldots, w_4$ are the tunable weights of the sum, $H$ is the entropy of the distribution computed over the duration of the notes in the score and $\sigma$ is the standard deviation of such distribution  \cite{eerola2003}, $ED$ is the Event Density as defined in Equation \ref{eq:LLFs:ED}. $A$ measures the synchrony of phenomenal accents in the part and in the metrical hierarchy. The metrical hierarchy of phenomenal accents refers to the natural decomposition of a rhythmic patter into strong and weak accents performed by a musician and induced by the sheet music, e.g., the strong accents in the first beat of the bars. Musicians are used to this metrical hierarchy and it is easy for them to follow it. The phenomenal accents in the part refer to where the composer chose to place the accents in the part in order to induce a certain emotion or to express a specific expression. The misalignment between the two systems of accents, which is captured by $A$, adds some complexity in the execution for the musicians.
%
\section{A feature-based analysis scenario: Networked Music Performance}
%\chapter{Low-Level and Mid-Level feature analysis for Networked Music Performance}
\label{sec:NMP}
%\section{Introduction} \label{sec:NMP:introduction}

%We described the hand-crafted features as a representation of the signal domain that can be subsequently linked with the semantic domain. However, the hand-crafted features can be used alone to 

%in fact be employed in other application scenarios. In this section, we test the potential of a feature based analysis in the Networked Music Performance scenario. 
The feature representation of the signal domain is not only useful for the linking with the semantic domain, but it is also used for perform a manual analysis of problems that regard the musical content. As discussed in the previous Sections, in order to use such features, it is required to hold a full knowledge of the characteristics they are able to extract. In this Section, we discuss a research activity to develop a better understanding of the features and to demonstrate their use for a manual analysis for the task of Networked Music Performance. The Networked Music Performance (NMP) promises to revolutionize interactive music performances (e.g. remote rehearsals, music teaching) by allowing remote players to interact with each other from remote physical locations through an Internet connection. %Networked Music Performance is a mediated interactional modality with a tremendous potential impact on professional and amateur musicians, as it enables real-time interaction from remote locations.

%
%In the latest decade, the speed and diffusion of the Internet connections have experienced a tremendous improvement. This scenario has made possible the development of novel applications and paradigms such as the services to stream music directly from the Internet or softwares to listen to music with the tempo that matches the user's running pace.
%
%Among the many applications that are crucially dependent on the fast Internet connections,

One  of the known limiting factors of distributed networked performances is the impact of the unavoidable packet delay and jitter introduced by IP networks, which make it difficult to keep a stable tempo during the performance. Computer-based systems enabling music performance have been investigated starting from the \lq 70s \cite{barbosa2003displaced} and recently different network architectures have been proposed as enabling design paradigms for NMP systems, ranging from client-server \cite{saputra2012design,gu2005network} and master-slave \cite{renwick2012sourcenode} to decentralized peer-to-peer infrastructures \cite{stais2013networked,chafe2011living}. 

%The delay tolerance is typically estimated to be 20-30 ms \cite{carot2009fundamentals} (corresponding to the time that the sound field takes to cover a distance of 8-9 m), which has been shown to correspond to the maximum physical separation beyond which keeping a common tempo for rhythmic music interaction without conductor becomes difficult. However, the sensitivity to delay and the quality of the musical experience in the context of NMP is influenced by several additional factors \cite{Bouillot2007}. In \cite{barbosa2011influence}, the authors investigate the correlation between perceptual attack times of the instrument (i.e., the time that the instrument takes to reach its maximum loudness) and the sensitivity to network delay, concluding that instruments with a slow attack (e.g. strings) usually tolerate higher latency. In \cite{Chew2004}, the authors investigate the correlation between accepted network latency and the genres of pattern-based music. 

Other limiting factors of NMP depend on the performance itself, including the qualities of the involved instruments or of the performed music. In \cite{barbosa2011influence}, two classes of instruments are defined, with slow or fast attack times, and it is found that the use of instruments with a slow response helps the musicians to be less sensible to the latency issues. In \cite{Chew2004} some broad class of music genres are defined to investigate their correlation with the tolerance to the network latency. The above-mentioned studies conduct a qualitative analysis of the additional factors to investigate their correlation with the sensitivity to delay and the overall quality of the music performance. This qualitative analysis, however, does not provide information on the degree of the impact of the various factors to the final tolerance to the delay. A quantitative analysis, instead, can be more useful to better understand the issues raised by NMP and to potentially relax the latency constraints for the design of the network architecture. 

In this Section, we aim at conducting a quantitative study about the effects of some acoustic properties on the musicians' tolerance to the delay and on the quality of music performance. We perform the quantitative study by taking advantage of the dimensional feature-based representation of music analysis \cite{Kim2005,Zanoni2012}. We are interested in the objective qualities of the musical content, therefore the signal domain is formalized by means of a set of low-level and mid-level features. We analyze the NMPs by means of quality metrics, for which we extract the trend of the tempo kept during the performance and we collect annotations from the musicians about the perceptual quality of the musical interaction, and we investigate the quantitative correlation between the feature representation of the signal domain and the resulting quality of the performance by manually analyzing the resulting data. This research activity allows us to develop a better understanding of the low-level and mid-level features by investigating which factors affect the subjective and objective quality of a NMP. As a result of the activity, we are able to define which network constraints must be satisfied depending on the different properties of the music that is willing to be played.

In the following we provide an overview of the state of the art and theoretical background on NMP (Section \ref{sec:NMP:background}). In order to conduct this analysis, we implemented a testbed for psycho-acoustic analysis emulating the behavior of a real IP network in terms of variable transmission delay and jitter, and we recorded a set of NMPs with different musicians, instruments, songs and tempo within the song. We present the details on the setup of the testbed in Section \ref{sec:NMP:testbed}. The obtained recording are processed in order to extract the features for the signal domain and the quality metrics of the performances. We describe the extraction techniques in Section 
\ref{sec:NMP:domain}. Finally, in Sections \ref{sec:NMP:qualResults} and \ref{sec:NMP:quantResults} we present the results of the manual analysis of the collected data and we draw some final considerations in Section \ref{sec:NMP:conclusions}.

In our study we conducted a set of experiments with both male and female musicians. In the following discussion, therefore, we use the \textit{singular they} with the gender-neutral meaning. 

%such as the rhythmic complexity of the performed piece, the timbral characteristics of the instruments and the type of musical part that is being performed (e.g. melody, chord comping, sustained harmony) has not yet been proposed in the literature. 


\subsection{Background}\label{sec:NMP:background}
In order to reproduce realistic environmental conditions for NMP, several technical, psycho-cognitive and musicological issues must be addressed. In particular, at network level, very strict requirements in terms of latency and jitter must be satisfied to keep the one-way end-to-end transmission delay below a few tens of milliseconds. 

The overall delay experienced by the players includes multiple contributions due to different stages of the audio signal transmission: the first is the processing delay introduced by the audio acquisition, processing, and packetization; the second is the pure propagation delay over the physical transmission medium; the third is the data processing delay introduced by the intermediate network nodes traversed by the audio data along their path from source to destination, the fourth is the playout buffering which might be required to compensate the effects of jitter in order to provide sufficiently low packet losses to ensure a target audio quality level.

Some preliminary studies on the delay tolerance for live musical interactions have already appeared: in \cite{gurevich2004simulation,chafe2010effect,chafe2004effect,chafe2004network} the authors evaluated the trend of tempo variations (measured in Beats Per Minute - BPM) while performing predefined rhythmic patterns through hand clapping, in different latency conditions. A similar analysis was integrated with an evaluation of the musicians' subjective rating of the performance quality in \cite{carot2009towards}. In \cite{barbosa2011influence}, the authors show that the human auditory system focuses on onsets produced by instruments with a short or almost impulsive attack time, whereas it tends to perceive less immediately those onsets associated to instruments with a slow attack. The impact of delay on the synchronism of the performance is therefore expected to be more clearly perceivable when using musical instruments with a fast attack, rather than when using instruments with a slower attack time. This means that the choice of musical instrument are informative in presence of network delay. In practice, however, musicians tend to adjust their playing technique according to the specific attack time of the instrument played. For example, organ players are used to naturally compensating the delay elapsed between the pressure of the keyboard keys and the sound emission from the pipes, as well as the time that the sound takes to travel back from the pipes to the musician. To a smaller extent this is also true for piano players. In this case the delay between pressing a key and detecting the corresponding note onset varies between $30$ and $100$ ms, depending on sound loudness and musical articulation (e.g. \textit{legato, staccato}) \cite{askenfelt}. For some categories of instruments, it has been shown that the expressive intention and direction of the musician (i.e., subjective artistic and interpretation choices, which are in turn affected by a particular emotional state during the performance) can have a significant impact on sonological parameters such as attack, sustain and decay time \cite{clarinet}. In our study we do not evaluate the impact of the instrument attack time on the performance interplay, and consider this attack time simply as part of the overall delay perceived by the musician. 

One specific aspect that has been addressed in the literature is the role played by a musical instrument in a performance. In western music, some instruments have a more pronounced ``leading'' role than others. In \cite{Carot07networkmusic} the authors define two main musical roles, the \textit{solo} and the \textit{rhythmic} role, and they identify some approaches of musical interaction that depend on the network delay that is set. The best situation is called \textit{Realistic Interaction Approach} (RIA), when the network delay is lower than 25 ms and both roles can play with total interaction as if they were playing in the same physical place. Beyond the 25 ms threshold, the performance usually shifts to a \textit{Master-Slave Approach} (MSA), where the rhythmic part leads the interaction with the solo part, as the Master of the performance, and the solo part follows the provided tempo, acting as the Slave. The rhythmic players are therefore expected to keep a steady tempo even when the other musicians are playing off-tempo. From the solo perspective, the synchronization is good, while the interaction is more difficult. When the delay approaches the 50 ms value, \cite{Carot07networkmusic} suggests a behavior named \textit{Laid Back Approach} (LBA), whereby the solo plays slightly behind the groove led by the rhythmic part. This approach is a common solo style in jazz music and produces an acceptable interaction for the musicians even for high delay. An alternative MSA approach is the \textit{Delayed Feedback Approach} (DFA), where the rhythmic part holds its leading role, while receiving their own sound feedback with a slight delay. If the delay of the feedback is similar to the network delay, the musician who is playing the rhythmic role will listen to their feedback in sync with the solo part. In DFA, the solo musician perceives a synchronization quality similar to the MSA, while it is more challenging for the rhythmic musician to handle the delayed feedback. 

In our study we extend the set of roles to four, where we split the rhythmic parts into chord comping and drums, and we include the sustained harmony. For this reason, we do not use the taxonomy of approaches defined by \cite{Carot07networkmusic}.


\subsection{Testbed setup}\label{sec:NMP:testbed}
\begin{figure}[!tb]
  \centering
  \includegraphics[width=.95\textwidth]{img/NMP/setup}
  \caption{Testbed setup}
 \label{fig:NMP:testbed}
\end{figure}
In order to analyze the correlation of the music played with the tolerance to delay, we implemented a testbed setup to emulate the NMP scenario. Our experiments involved pairs of musicians playing in two phono-insulated near-anechoic rooms (sound rooms), to avoid any audio feedback or visual contact, as depicted in Figure \ref{fig:NMP:testbed}. Visual contact was provided not even by means of video streaming because video processing time is larger than audio processing time and would have increased the minimum achievable end-to-end delay. The musicians were also forbidden to verbally communicate to their counterpart during the performance. Each room was equipped with a desktop PC running the Soundjack software, which is a peer-to-peer publicly available software \cite{carot2008distributed} that also implements a direct real-time evaluation of the experienced one-way end-to-end latency (thus including processing, buffering and playout delays).  Each PC was connected to an external sound card via high-speed connection (FireWire and AES/EBU) operating at a sampling rate of 48 kHz. The sound card was connected to high-quality headphones and microphones. An additional PC (with two network interfaces) running the WANem network emulator \cite{wanem} was placed in between. 
Through the WANem emulator we could manually select the delay and jitter of the emulated network, in order to replicate the NMP real-world scenario. The network interfaces of the three PCs were connected to each other through a Fast Ethernet switch. The PCs of the sound rooms were configured to communicate exclusively through the interfaces of the WANem emulator, thus preventing any direct communication between them. 

Each musician was able to hear their own instrument as well as the instrument of the other player through headphones. The choice of the headphones was made in order to minimize the possible delay and to avoid loop-feedback effects. The two audio signals were transmitted through the Local Area Network of the building. During the experiments, all the involved LAN segments were free of other traffic. The audio tracks were recorded as follows: the audio data generated by \textit{performer A} were recorded directly after the electric transduction of the microphone, whereas the audio data generated by \textit{performer B} were recorded from the SoundJack feedback after propagation of the audio stream through the network, i.e. as heard through \textit{performer A}'s headsets.

%\subsection{Scores and Network Parameters}
We considered three different music pieces: \textit{Yellow Submarine} (by The Beatles) at different values of BPM (88,110,130), \textit{Bolero} (by Maurice Ravel), and \textit{Master Blaster} (by Stevie Wonder), arranged for four different parts: main melody (M), chord comping (CC), sustained harmony (SH), and drums (D).  Scores were released to the musicians in advance.

\begin{table}[tb]
  \caption{Combination of parts played in each experiment session. M: main melody; CC: chord comping; SH: sustained harmony; D: drums}
  \centering %\small
  \label{tab:NMP:sessions}
  \bgroup
  \def\arraystretch{1.5}
 \begin{tabular}{||c|c|c|c|c||}
 \hline
 \hline
  Id & Instrument A& Part A & Instrument B & Part B \\
 \hline
 \hline
  1 & Acoustic Guitar & M & Classic Guitar & CC\\
  2 & Electric Piano & M & Drums & D\\
  3 & Keyboard (strings)  & SH & Drums & D\\
  4 & Keyboard (strings)  & SH & Electric Guitar & CC\\
  5 & Clarinet & M & Clarinet & M\\
  6 & Eletric Guitar  & CC & Drums & D\\
  7 & Keyboard (strings)  & SH & Clarinet & M\\
 \hline
 \hline
   \end{tabular}
   \egroup
\end{table}

Our experiments involved 8 musicians with at least 8 years of musical experience, all with semi-professional or professional training level, and 7 instruments, i.e.:
Acoustic, Classic and Electric Guitar, Clarinet, Drums, Electric Piano and Keyboards with strings samples. Each musician played at least one of the instruments. The musicians were grouped in 7 pairs according to the combinations listed in Table \ref{tab:NMP:sessions}. Note that some musicians performed in more than one pair (e.g., one clarinetist performed twice, i.e. in pairs 5 and 7, whereas the pianist played electric piano and keyboard in pairs 2, 3, 4 and 7). For a given pair, each musician performed only one of the four parts for each of the three considered musical pieces, as detailed in Table \ref{tab:NMP:sessions}. Musicians in pairs 5 and 6 had regularly performed together in the last years, where the remaining pairs had never played together before. However, in order to avoid biases due to prior common performances, all the pairs were allowed to practice together in the testbed environment until they felt sufficiently confident. Before participating to our experiments, none of the players had ever experienced networked music interactions.

\begin{table}[tb]
  \caption{Tested network parameters and tempo settings}
  \centering %\small
  \label{tab:NMP:param}  
  \bgroup
  \def\arraystretch{1.5}
  \begin{tabular}{||c|c|c|c||}
\hline
\hline
 Piece & $\delta$ [BPM]& $\mu$ [ms]& $\sigma$ [ms]\\
\hline
\hline
 Yellow Submarine & 88,110,132 & 20,30,40,50,60 & 1 \\
 Bolero & 68  & 20,30,40,50,60 & 1 \\
 Master Blaster & 132  & 20,30,40,50,60 & 1 \\
\hline
\hline
  \end{tabular}
  \egroup
\end{table}

The recording procedure was repeated several times for each piece. As reported in Table \ref{tab:NMP:param}, each recording was characterized by different tempo and network settings in terms of reference BPM ($\delta$), network latency and jitter. The two latter parameters were set by assigning each IP packet a random delay $T_{net}$, statistically characterized by independent identically distributed Gaussian random variables with mean $\mu$ and standard deviation $\sigma$. The payload of each packet contained $128$ 16-bit-long audio samples, corresponding to a duration of $2.67$ ms. For the considered values of $\mu$ and $\sigma$, we set the receiver buffer size to 4 packets (i.e. $512$ audio samples) and measured the number of buffer overruns/underruns during each recording. The overall probability of overrun/underrun events turned out to be smaller than $1\%$. This value is representative of realistic traffic conditions of a telecommunication network. Note that overruns/underruns generate glitches (e.i, distortions in the reproduction of the received audio signal) which affect the overall audio quality perceived by the musicians. 
 

Note also that, as $T_{net}$ accounts only for the emulated network delay, the additional latency $T_{proc}$ introduced by the audio acquisition and the audio rendering processes must be taken into account in the computation of the one-way overall delay time $T_{tot}=T_{net}+ T_{proc}$.
More specifically, the processing time $T_{proc}$ includes: in-air sound propagation from instrument to microphone; transduction from the acoustic wave to electric signal in the microphone; signal transmission through the microphone's wire; analog to digital conversion of the sender's sound card, internal data buffering of the sound card; processing time of the sender's PC to packetize the audio data prior to transmission; processing time of the receiver's PC to depacketize the received audio data; queuing time of the audio data in the application buffer at the receiver side; digital to analog conversion of the receiver's sound card; transmission of the signal through the headphone's electric wire; transduction from electric signal to acoustic wave in the headphones.
We experimentally evaluated $T_{proc}$ by measuring the end-to-end delay $T_{tot}$ when setting $\mu=0$ and $\sigma=0$, i.e., $T_{net}=0$. The measured time was $T_{proc}=15$ ms, which is larger than the one reported in \cite{carot2007networked}. This is mainly due to the use of generic sound card drivers, which increased the processing time of SoundJack.

During each recording session, the order of the proposed network configurations was randomly chosen and was kept undisclosed to the testers, in order to avoid biasing or conditioning. Two measures of metronome beats at the reference BPM were played before the beginning of each performance. At the end of each performance, the testers were asked to express a rating on the quality of their interactive performance. The details on such quality are provided in the next Section.

%\subsection{Collection of the signal domain and the semantic codomain}\label{sec:NMP:domain}
%We formalize the signal domain by extracting rhythmic and timbral features from the recordings of the performance and from the score of the parts. The quality of performances is evaluated by means of subjective metrics, asked to the musicians after each performance, and of objective metrics, through a processing technique from the recordings.


\subsection{The feature representation of the signal domain}\label{sec:NMP:domain}

\begin{table}[tb]
  \caption{Timbral characterization for each instrument}
  \centering %\small
  \label{tab:NMP:instruments}
  \bgroup
  \def\arraystretch{1.5}
\begin{tabular}{||c|c|c|c|c|c|c||}
 \hline
 \hline
 Instrument  & $SC$ & $SSp$ & $SSk$ & $SK$ & $SF$ & $SE$ \\
 \hline
 \hline
Ac. Guitar &   2047 & 4109 & 2.76 & 10.25 & 0.19 & 0.76 \\
Clarinet & 1686 & 2272 & 4.85 & 31.81 & 0.07 & 0.731 \\
Cl. Guitar &   3263 & 4680 & 1.57 & 4.43 & 0.22 & 0.841 \\
Drums & 7903 & 7289 & 0.35 & 1.57 & 0.61 & 0.936 \\
El. Guitar & 1848 & 2522 & 3.70 & 23.46 & 0.09 & 0.818 \\
El. Piano & 2101 & 4251 & 3.16 & 12.26 & 0.16 & 0.734 \\
Keyboard & 1655 & 3065 & 4.39 & 23.77 & 0.1 & 0.733 \\
 \hline
 \hline
   \end{tabular}
   \egroup
\end{table}

We formalize the signal domain by extracting timbral and rhythmic descriptors from the recordings of the performance and from the score or symbolic representation of the parts, respectively. 

As far as the timbral features are regarded, we consider the timbre as a property of the instruments and we do not track their evolution during the performances. For each instrument, we compose an audio file with a representative selection of recordings of its timbre. For example, the timbral characterization of the drums included the recording of each percussive instrument from the drum set, whereas the characterization of guitar included different kinds of playing techniques, like chords played as arpeggio and plucked strings. From the representative audio files we then extract the following features: Spectral Centroid ($SC$), Spectral Spreadness ($SSp$), Spectral Skewness ($SSk$), Spectral Kurtosis ($SK$), Spectral Flatness ($SF$) and Spectral Entropy ($SE$). The definition and interpretation of such features was provided in Section \ref{sec:LLFs:LLFs}. It is worth remembering that the Spectral Centroid provides an indicator of the brightness of the timbre, while Spectral Entropy and Flatness can be interpreted as an estimate of the noisiness. 

The features are extracted by means of the MIRToolbox \cite{Lartillot2007} and their average values computed over each audio file are reported in Table \ref{tab:NMP:instruments}. We can notice that the Drums exhibit the highest values of Flatness and Entropy, hence noisiness, and a rather high Spectral Centroid. On the other side, the Clarinet presents the lowest value of noisiness and a low Spectral Centroid, given by its warm deep timbre. Among the guitars, we do not use distortion effect for the Electric Guitar, hence its timbre is rather clean and has a lower Spectral Flatness than the Acoustic Guitar, while showing a higher Entropy.

\begin{table}[tb]
  \caption{Rythmic characterization of the musical pieces performed during the tests}
  \centering %\small
  \label{tab:NMP:pieces}  
  \bgroup
  \def\arraystretch{1.5}
 \begin{tabular}{||c|c|p{1.4cm}|p{1.4cm}|p{1.8cm}|p{1.8cm}|p{1.8cm}||}
 \hline
 \hline
Part &  Feature & Bolero & Master Blaster & Yellow Submarine (88 BPM) & Yellow Submarine (110 BPM) & Yellow Submarine (130 BPM)\\
 \hline
 \hline
 \multirow{2}{*}{M}& ED & 2.1407 & 2.1667 & 1.5253 & 1.9067 & 2.2880 \\
           & RC & 5.5337 & 5.5627 & 5.4160 & 5.7094 & 6.0567 \\
\hline
\multirow{2}{*}{CC}& ED & 1.3222 & 2.6542 & 1.8333 & 2.2917 & 2.7500 \\
          & RC & 3.4516 & 6.8903 & 5.3064 & 5.6455 & 5.9592\\
\hline
\multirow{2}{*}{SH}& ED & 0.3778 & 0.5778 & 0.8213 & 1.0267 & 1.2320 \\
          & RC & 2.9364 & 5.3444 & 3.8062 & 4.0208 & 4.2237\\
\hline
\multirow{2}{*}{D}& ED & 2.0148 & 4.3514 & 1.5253 & 1.9067 & 2.2880 \\
         & RC & 6.0285 & 5.7255 & 4.5228 & 4.7548 &  4.9767 \\
 \hline
 \hline
   \end{tabular}
   \egroup
\end{table}

As far as the rhytmic features are regarded, we extract some of the MLFs described in Section \ref{sec:LLFs:MLFs}, i.e., the tempo, the Rhythmic Complexity and the Event Onset. As mentioned, the tempo was fixed and provided to the musician, and a particular song, \textit{Yellow Submarine}, was executed at different tempi. We manually compute the Event Density \cite{Lartillot2007} from the sheet music of the parts and we extract the Rhytmic Complexity \cite{povel} using the MIDI Toolbox \cite{Eerola2004}. With reference to the Equation \ref{eq:RC}, we use the weights defined in the MIDI Toolbox $w_1= 0.7$, $w_2=0.2,$  $w_3=0.5$, $w_4=0.5$, i.e., the Entropy of the distribution of the notes has the greatest relevance, then the standard deviation of the distribution of the notes and the the synchrony of phenomenal accent in the score are equally weighted and finally, the Event Density receives a low weight. 

The rhythmic features for each part are summarized in Table 
\ref{tab:NMP:pieces}. We notice that the Sustained Harmony has generally low values of Event Density and Rhythmic Complexity, since it usually involves one onset every measure. On the other side, the Chord Comping of \textit{Master Blaster} and the drums in \textit{Bolero} are rhythmically challenging, due to the variation between even and odd patterns. Moreover, since the Event Density depends on the tempo and Rhythmic Complexity depends on Event Density, we notice that increasing the BPM of \textit{Yellow Submarine} produces an increase of both features.




%in this study we provide an evaluation of the impact of network conditions on the quality of the musical experience, according to the type of the instruments and to some characteristics of the performance. As far as the type of instrument is concerned we adopt a timbral feature-based representation, whereas we exploit musical part, Event Density \cite{Lartillot2007} and Rhythmic Complexity \cite{povel} of the performed pieces to characterize the performance. 

\subsection{The quality metrics}\label{sec:NMP:codomain}
Several perceptual and musicological aspects affect the perception of the overall quality of a performance and therefore it is not clear how to measure such quality. Musicians are the ideal candidates to estimate it, since they usually evaluate their own performance, during rehersals, in order to improve their skills. For this reason, we define two subjective metrics annotated directly by the musicians involved in the performance. Such subjective metrics are however affected by the personal bias of the musicians: two musicians might differently rate the overall quality of the same performance, according to their experience or the importance they assign to different aspects. In order to ease this subjectivity issues, we also define an objective metric, which is unbiased from the musicians' opinion, by computing the trend of the tempo over the performance.

At the end of each of the recordings, we asked to the musicians to rate the quality of the performance. The musicians provided two annotations: one about the quality of the interaction and one about the perception of the delay. The former is defined as the perceived quality $Q_{perc}$, and is annotated within a five-valued range, from $Q_{perc}=1$, meaning the performance was very poor, to $Q_{perc}=5$, i.e., very good. The metric $Q_{perc}$ is not related to the audio quality experienced by the musicians, but only to the evaluation of the overall satisfaction of their experience and interaction with the counterpart. The latter is defined as the perceived network delay, $D_{perc}$, and it ranges between $1$, i.e., the delay was intolerable, and $4$, i.e., the delay was not perceivable. In case the players spontaneously aborted their performance within the first $50$ seconds, $D_{perc}$ was set to $1$ and $Q_{perc}$ was set to $0$ by default. 

After the recording sessions, we extract the objective quality metric as the trend of the tempo during the performances, i.e., the tendency to slow down or accelerate in the initial part of the performance. For the first $50$ seconds we compute a linear regression of the BPM over the sparse BPM measurements. In order to do so, we first manually annotate the beats of the performance, as it is indicated in the music score. We estimate the beats occurring in the silences as equally distant from the surrounding beats. We are able to compute the instantaneous BPM from the time difference between two consecutive beats. We compute the BPM trend by smoothing the sequence of instantaneous BPM. The audio tracks are divided in $N=20$ time windows, each lasting $5$~s, with a 50\% overlap ($2.5$~s). For each time window, we compute the average BPM as $b(t_n)$ with $t_n=n*2.5$s and $n=1,...,N$. We estimate the tendency of the performance to accelerate or decelerate by considering the slope of the linear approximation of the BPM trend. We estimate the intercept $\beta$ and the slope $\kappa$ with linear interpolation, 
\begin{equation}
\argmin{\kappa, \beta} \frac{1}{N} \sum_{n=1}^{N} \left(b(t_n)-(\kappa t_n + \beta)\right)^2.
\end{equation}
In our experiments, the average Mean Square Error of the linear approximation is about $1.75$\%, i.e., it is accurate to approximate the BPM trend as a first order polynomial.

We consider the slope $\kappa$ as the objective metric for the evaluation of the performance quality: $\kappa=0$ means steady tempo; $\kappa>0$ means that the musician is accelerating; $\kappa<0$ means that the musician is slowing down and thus is unable to keep up with the tempo. 
%
%\subsection{Numerical results}\label{sec:NMP:results}
%In this Section, we build the linking function by means of a manual data analysis. We first provide some general and qualitative considerations on the BPM trend of the performance, and then we discuss the quantitative analysis of the NMP.

\subsection{Qualitative evaluation}\label{sec:NMP:qualResults}
\begin{wrapfigure}{R}{0.3\textwidth}
  \begin{center}
     \includegraphics[width=0.2\textwidth]{img/NMP/legend_smaller}
  \end{center}
  \caption{Legend for Figures \ref{fig:NMP:melody} and \ref{fig:NMP:drums} }
 \label{fig:NMP:legend} 
\end{wrapfigure}

We first provide some qualitative comments on the trend of the BPM curve $b(t_n)$ extracted from the execution of \textit{ Yellow Submarine} for different combinations of instruments and parts, various values of $T_{tot}$ in the range between $15$ and $75$ ms and three different values of $\delta$ (as reported in Table \ref{tab:NMP:param}). The lower bound of the tested delay values (i.e. $T_{tot}=15$ ms) is obtained by setting $T_{net}=0$ ms, meaning that no network delay is added to the unavoidable processing time $T_{proc}$. For values of $T_{tot}$ above $75$ ms (i.e., when $T_{net}=60$ ms), a considerable amount of executions were aborted by the musicians due to the extreme difficulty in maintaining synchronization. Therefore, we limit our analysis to delay ranges which allowed every pair of musician to perform the piece uninterruptedly for at least one minute.
The results are reported in Figures \ref{fig:NMP:melody} and \ref{fig:NMP:drums}, while the legend is shown in Figure  \ref{fig:NMP:legend}. The different colors identify the amount of total network delay $T_{tot}$ applied, while the style of the lines identifies the nominal BPM. The results show that in all the considered recordings an initial deceleration occurs in the first few seconds, when the players adjust their tempo until they find a balance, allowing them to reach the required degree of synchronization. Such initial deceleration is nearly absent for small network end-to-end delays and reference BPM, but it becomes much more pronounced for large values of $T_{tot}$ and $\delta$. In particular, the scenario with $\delta=132$ BPM and $T_{tot}=75$ ms presents an initial tempo reduction of 12-20 BPM in all the tested combinations of instruments and parts.
In addition, as shown in Figure \ref{fig:NMP:melody}, combining typically non-homorhythmic parts such as Melody (M) and Chord Comping (CC) or M and Sustained Harmony (SH) leads either to a tendency to constantly decelerate (see Figure~\ref{fig:NMP:melody}, left-hand side), which is more pronounced for large $\delta$ , or to a \lq\lq saw tooth'' pattern in which the players periodically try to compensate the tempo reduction (Figure \ref{fig:NMP:melody}, middle). Note that, in the latter case, there is no such pattern in the benchmark scenarios with $T_{tot}=15$ ms. The difference in the behavior of SH and CC when interacting with M is also due to the type of rhythmic interplay that takes place. Chord Comping, in fact, tends to closely follow and match the tempo of the Melody, while Sustained Harmony is a steady accompaniment (``pad") with more relaxed on-time constraints. As M is expected to meander off-tempo, it is harder for SH and M to stay in sync, and adjustments happen in bursts.


\begin{figure}[!tb]
  \centering
  \includegraphics[width=\textwidth]{img/NMP/fig1_wider}
  \caption{BPM trend over time when playing \textit{Yellow Submarine}, for different combinations of parts, instruments,  end-to-end delays $T_{tot}$ (identified by the color) and reference BPM $\delta$, identified by the type of line (solid, dashed, dotted).}
%  (\textit{performer A} on top, \textit{performer B} on bottom), for various values of
   \label{fig:NMP:melody} 
\end{figure}
\begin{figure}[!tb]
  \centering
  \includegraphics[width=\textwidth]{img/NMP/fig2_wider}
  \caption{BPM trend over time when playing \textit{Yellow Submarine} with Drums, combined with different parts and instruments, for various values of end-to-end delays $T_{tot}$ (identified by the color) and reference BPM $\delta$, identified by the type of line (solid, dashed, dotted).}
 \label{fig:NMP:drums} 
\end{figure}


%
%\begin{figure}[!tb]
%  \centering
%  \includegraphics[width=0.2\textwidth]{img/NMP/legend_smaller}
%  \caption{Legend for Figures \ref{fig:NMP:melody} and \ref{fig:NMP:drums} }
% \label{fig:NMP:legend} 
%\end{figure}


When two homo-rhyrthmic parts (those that are expected to keep a steady tempo, such as CC and SH) are combined, $b(t_n)$ tends to remain almost constant (see Figure \ref{fig:NMP:melody}, on the right-hand side, where a slight negative slope occurs only at $\delta=132$ BPM).
A similar behavior is observed when M, CC or SH combine with Drums (See Figure \ref{fig:NMP:drums}), despite the fact that the two parts are not always homo-rhythmic. This is due to the fact that drums tend to have a very specific rhythmic ``leading role" in western music, therefore the other musicians generally tend to follow the drummer.

Based on the above results, we conclude that the choice of the combination of instruments and parts has a significant impact on the capability of the musicians to keep a steady tempo. 
In the next Section, we will give a more in-depth analysis of the impact of single rhythmic and timbral features characterizing the specific combination of parts and instruments on the subjective and objective performance quality metrics.

\subsection{Quantitative evaluation}\label{sec:NMP:quantResults}
We now analyze the impact of different end-to-end delays $T_{tot}$ on the subjective quality metric $D_{perc}$ and on the BPM slope $\kappa$, for various values of the rythmic and timbral features described in Section \ref{sec:NMP:domain}. The interaction quality rating $Q_{perc}$ resulted to be strongly correlated to $D_{perc}$, therefore for the sake of brevity we do not report such results.


%\subsubsection{Dependency of Quality Metrics on Rythmic Features}
\begin{figure}[!tb]
\begin{flushright}
  \subfloat[Subjective Perception of Delay $D_{perc}$]{\includegraphics[width=.77\textwidth]{img/NMP/minRC_PD}\label{fig:NMP:minRC_SubjPerc}}  \hfil
  \subfloat[Average BPM Linear Slope $\kappa$]{\includegraphics[width=.72\textwidth]{img/NMP/minRC_BPM}\label{fig:NMP:minRC_LinSlope}}        
\end{flushright}
\caption{Dependence of $\kappa$ and $D_{perc}$ on the minimum Rythmic Complexity $RC$ for different values of $T_{tot}$.}
\label{fig:NMP:minRC}
\end{figure}


 

For every recording, we consider the maximum and minimum values of each feature among the two parts and instruments played by the musicians. For example, in test session 2 (see Table \ref{tab:NMP:sessions}) when performing \textit{Bolero}, the minimum Event Density ($ED$) is 2.01 ($ED$ of the D part) and the maximum $ED$ is 2.14 ($ED$ of the M part). Conversely, the minimum Rhythmic Complexity ($RC$) is 5.53 (on the M part) and the maximum $RC$ is 6.03 (on the D part).

\begin{figure}[!tb]
\begin{flushright}
  \subfloat[Subjective Perception of Delay $D_{perc}$]{\includegraphics[width=.77\textwidth]{img/NMP/minED_PD}\label{fig:NMP:minED_SubjPerc}}  \hfil
  \subfloat[Average BPM Linear Slope $\kappa$]{\includegraphics[width=.72\textwidth]{img/NMP/minED_BPM}\label{fig:NMP:minED_LinSlope}}        
\end{flushright}
\caption{Dependence of $\kappa$ and $D_{perc}$ on the minimum Event Density $ED$ for different values of $T_{tot}$}
\label{fig:NMP:minED}
\end{figure}

Figure \ref{fig:NMP:minRC_SubjPerc} reports the subjective delay perception $D_{perc}$ attributed by the pairs of testers to their performances, for different values of $T_{tot}$, as a function of the minimum $RC$ between the two parts. For the sake of clarity, only four values of $T_{tot}$ are reported, where $T_{tot}=T_{proc}=15$ ms is considered as benchmark. 

\begin{figure}[!tb]
\begin{flushright}
  \subfloat[Subjective Perception of Delay $D_{perc}$]{\includegraphics[width=.77\textwidth]{img/NMP/minSE_PD}\label{fig:NMP:minSE_SubjPerc}}  \hfil
  \subfloat[Average BPM Linear Slope $\kappa$]{\includegraphics[width=.72\textwidth]{img/NMP/minSE_BPM}\label{fig:NMP:minSE_LinSlope}}        
\end{flushright}
\caption{Dependence of $\kappa$ and $D_{perc}$ on the minimum Spectral Entropy $SE$ for different values of $T_{tot}$}
\label{fig:NMP:minSE}
\end{figure}


Results shows that, for a given value of minimum $RC$, the average $D_{perc}$ decreases when $T_{tot}$ increases. Moreover, for a given $T_{tot}$, increasing $RC$ also has a negative impact on the average quality rating. However, the reduction of $D_{perc}$ is more relevant for large values of $T_{tot}$. 
%Similar results are obtained when considering the subjective delay perception $D_{perc}$, averaged over the pair of performers (see Figure \ref{}).

\begin{figure}[!tb]
\begin{flushright}
  \subfloat[Subjective Perception of Delay $D_{perc}$]{\includegraphics[width=.77\textwidth]{img/NMP/minSF_PD}\label{fig:NMP:minSF_SubjPerc}}  \hfil
  \subfloat[Average BPM Linear Slope $\kappa$]{\includegraphics[width=.72\textwidth]{img/NMP/minSF_BPM}\label{fig:NMP:minSF_LinSlope}}        
\end{flushright}
\caption{Dependence of $\kappa$ and $D_{perc}$ on the minimum Spectral Flatness $SF$ for different values of $T_{tot}$}
\label{fig:NMP:minSF}
\end{figure}

We now analyze how the average BPM linear slope $\kappa$ is affected by the minimum rhythmic complexity $RC$. For large values of $RC$ (see Figure \ref{fig:NMP:minRC_LinSlope}), we found slightly negative values of $\kappa$ (which denote a tendency to slow down) even in the benchmark scenario. As expected, the need of synchronism increases when musicians are playing more complex parts and the lack of typical synchronization cues, such as eye-contact, affects the performance even in absence of network delay. However, negative slopes tend to become much steeper for large values of $T_{tot}$, which suggests that the tolerance to the delay decreases for more complex musical pieces.

\begin{figure}[!tb]
\begin{flushright}
  \subfloat[Subjective Perception of Delay $D_{perc}$]{\includegraphics[width=.77\textwidth]{img/NMP/minSSk_PD}\label{fig:NMP:minSSk_SubjPerc}}  \hfil
  \subfloat[Average BPM Linear Slope $\kappa$]{\includegraphics[width=.72\textwidth]{img/NMP/minSSk_BPM}\label{fig:NMP:minSSk_LinSlope}}        
\end{flushright}
\caption{Dependence of $\kappa$ and $D_{perc}$ on the minimum Spectral Skewness $SSk$ for different values of $T_{tot}$.}
\label{fig:NMP:minSSk}
\end{figure}

Similar conclusions can be drawn on the dependence of the perceived delay $D_{perc}$ and the objective metrics $\kappa$ on the minimum $ED$, as depicted in Figure \ref{fig:NMP:minED}, due to the non-negligible correlation that exists between $RC$ and $ED$. 
These conclusions remain substantially unvaried if, instead of considering the minimum values of $RC$ and $ED$, we consider the maxima.
%\subsubsection{Dependency of Quality Metrics on Timbral Features}

\begin{figure}[!tb]
\begin{flushright}
  \subfloat[Subjective Perception of Delay $D_{perc}$]{\includegraphics[width=.77\textwidth]{img/NMP/minSC_PD}\label{fig:NMP:minSC_SubjPerc}}  \hfil
  \subfloat[Average BPM Linear Slope $\kappa$]{\includegraphics[width=.72\textwidth]{img/NMP/minSC_BPM}\label{fig:NMP:minSC_LinSlope}}        
\end{flushright}
\caption{Dependence of $\kappa$ and $D_{perc}$ on the minimum Spectral Centroid $SC$ for different values of $T_{tot}$.}
\label{fig:NMP:minSC}
\end{figure}



As far as timbral features are concerned, we observe that the noisiness of the instrument, which is captured by Spectral Entropy, Flatness and Spread, has a relevant impact on the perceived delay $D_{perc}$. For example, in Figures \ref{fig:NMP:minSE} and \ref{fig:NMP:minSF} we show $D_{perc}$ and $\kappa$ are affected by Spectral Entropy ($SE$) and Spectral Flatness ($SF$). We consider the minimum Entropy and minimum Flatness between the two involved instruments. Focusing on the objective metric $\kappa$ (see Figures \ref{fig:NMP:minSE_LinSlope} and \ref{fig:NMP:minSF_LinSlope}), we notice that as the $SE$ and the $SF$ increase, the tempo slowdown becomes more relevant. This impact is negligible for low network delays, but it grows significantly for fairly large values of $T_{tot}$. Similar considerations are valid for $D_{perc}$, as reported in Figure \ref{fig:NMP:minSE_SubjPerc}. Analogous findings also apply to the dependency of the quality metrics on the Spectral Spread and here not reported for the sake of brevity.  

Conversely, when considering the impact of Spectral Skewness ($SSk$) and Spectral Kurtosis ($SK$) on the performance metrics, we notice that, for a given delay $T_{tot}$, a change in their values does not cause the quality to perceivably worsen (see Figure \ref{fig:NMP:minSSk}, results on $SK$ not reported for conciseness). 

Finally, when looking at the influence of the Spectral Centroid $SC$ (i.e., of sound brightness) on the subjective quality metrics, results reported in Figure \ref{fig:NMP:minSC_SubjPerc} show that the perceptual metric $D_{perc}$ does not exhibit significant fluctuations due to a varying $SC$. However, for large values of $SC$, a slight tendency to decelerate emerges in Figure \ref{fig:NMP:minSC_LinSlope}, which shows the impact of $SC$ on the objective quality metric $\kappa$.

It is also worth noticing that $D_{perc}$ is not necessarily an indicator of quality degradation of the performance, but only on the musicians' subjective perception of the end-to-end delay. However, results reported in Figures \ref{fig:NMP:minRC_SubjPerc}, \ref{fig:NMP:minED_SubjPerc}, \ref{fig:NMP:minSE_SubjPerc}, \ref{fig:NMP:minSF_SubjPerc}, \ref{fig:NMP:minSSk_SubjPerc}, \ref{fig:NMP:minSC_SubjPerc}, show that such perception is strongly affected by the timbral and rhythmic characteristics of the combination of instruments and parts. For example, in Figure \ref{fig:NMP:minSSk_SubjPerc}, the perceived network delay $D_{perc}$ is larger for high values of $SSk$ and $T_{tot}$ than the value we would have in the case of low delays. This leads us to think that the musicians' capability of estimating the network delay is biased by the perceived interaction quality of the performance.
This means that large network delays (i.e., $T_{tot}\geq 75ms$) do not prevent networked musical interaction, but they limit the selection of the instrument/part combinations. Thus, the resulting experience can be satisfactory if the performer is willing to trade flexibility and perceived interaction quality with the convenience of playing over the network.

\subsection{Final considerations for the Networked Music Performance}\label{sec:NMP:conclusions}
In this Section, we have discussed the ability of LLFs and MLFs representing and analyzing the music signals in the context of Networked Music Performance.
%the effectiveness of LLFs and MLFs for the formalization of the signal domain, i..e, their ability of representing and analyzing the music signals. In order to develop a better understanding of the (low-level) semantics that is carried by these features, we conducted a research activity in the context of Networked Music Performance.
This particular activity has proven that the understanding of the hand-crafted features is crucial to address the problems that regard the signal domain.  We have in fact performed an extensive evaluation of the quality of NMPs as a function of numerous parameters, some concerning telecommunication network delays and conditions, others involving rhythmic and timbral descriptors of the musical instruments involved. The analysis also considers the influence of the role of the instrument on such quality metrics.

We have found that the possibility of enjoying an interactive networked musical performance is not only a function of the total network delay, but it also depends on the role and the timbral characteristics of the involved musical instruments, as well as the rhythmic complexity of the performance. When playing more rhythmically complex pieces, musicians exhibit a more pronounced tendency to decelerate for higher network latencies. Nonetheless, the rhythmical complexity does not significantly worsen their perception of the delay and of the interaction quality.
Among the timbral features, instruments with a higher Spectral Entropy and Spectral Flatness (such as guitars and drums) lead to larger tempo slowdown in case of higher network delays. In addition, they also amplify the negative impact of network delay on the perceived delay and interaction quality.

With these results in mind, we are able to estimate the network constraints for a NMP given the combination of parts and instruments that are performing remotely.

%
%The NMP is a promising application to allow musicians to jam and compose songs from different physical location through the network. The network connections introduce a delay that musicians must take into account. While several network architectures can be proposed in order to reduce such delay, it is important to understand which factors affect the musicians' tolerance to it and which conditions make it easy for musicians to remotely perform.
%In order to conduct this analysis, we implemented a testbed for psycho-acoustic tests, which emulates the behavior of a real telecommunication network in terms of variable transmission delay and jitter. %, and we quantitatively evaluated the impact of the various performance parameters on the trend of the tempo that the musicians were able to keep during the performance, as well as on the perceived quality of the musical interaction.
%We model the analysis of factors that affect NMP as a linking function between the low-level and mid-level interpretation of the signal domain and the semantics expressed by the perceived and objective quality of the performance. The former allows us to objectively analyze the role of rhythmic complexity in the parts, as well as the timbral properties of the instruments that are played. The latter provides both a subjective and an objective evaluation of the performance. We use a manual analysis of the correlation between the two domains to understand which factors of the music performance are involved.




\section{Learned Features}\label{sec:LLFs:learned}
The hand-crafted and model-based features have been widely used in the MIR literature to extract a representation of the signal domain by employing signal processing techniques, psychoacoustic and musicological models. The use of such features require to know in advance which characteristics are involved from the musical content and which features are able to extract such characteristics, which is not always feasible. In such situations, a typical approach is to extract all the possible features and evaluate their correlations with the desired output. However, it is possible that the features cannot extract the characteristics that matter for the specific issue and, therefore, it is necessary to identify a method to automatically extract a salient representation of the signal domain. In order to do so, we might want to mimic the human ability of organizing and elaborating the information from music.

% ----- The goal of Deep Learning -----
Recent neurological studies have shown that the human brain describes concepts (information) in hierarchical fashion \cite{Serre2007}. Indeed, the brain seems to process information through multiple stages of transformation and representation, providing multiple levels of abstraction \cite{Haykin1998}. The process is inducted by the physiological deep architecture of the mammal brain, where each level corresponds to a different area of cortex \cite{Serre2007}. As an example, in a simplistic form, while listening to a piano playing a melody, our brain first collects acoustic stimuli like frequency spectra and energy distributions. These pieces of information are then combined to build more complex concepts like the sequence of notes and the piano timbre. Notes, at the same time, are used to infer the concept of melody.

For this reason, the decomposition of decisional problems into sub-problems associated with different levels of abstraction results to be very effective \cite{Humphrey2013}. The analysis of the signal domain shall mimic this ability, but the extraction of hand-crafted features does not exploit enough depth \cite{Bengio2009}. This is why deep learning has been receiving great attentions in the last few years.

Deep learning techniques aim at emulating the human brain representation of information by providing several layers of analysis \cite{Bengio2009}. This is typically done by using a multi-layer structure such as neural networks. The input of the network represents the data under analysis (the spectrum of an audio frame in our case), and the output of each layer is a representation of the input (the \textit{learned} features in our case) obtained through processing (characteristic of the deep learning network). The more are the layers, the \textit{deeper} is the network and the higher is the reached level of abstraction.


\begin{figure}[t]
\captionsetup[subfigure]{justification=centering}
	\centering
%      \subfloat[Input]{\includegraphics[width=.45\textwidth]{img/Bootleg/original_input}\label{fig:LLFs:input}} \hfil
      \subfloat[Original spectrogram]{\includegraphics[trim=0cm 0cm 0cm 0cm,clip=true,totalheight=0.65\columnwidth]{img/Bootleg/original_input}\label{fig:LLFs:input}} \hfil
      \subfloat[Spectrogram reconstructed from the first layer]{\includegraphics[trim=2.9cm 0cm 0cm 0cm,clip=true,totalheight=0.65\columnwidth]{img/Bootleg/first_layer}\label{fig:LLFs:first}}
      \subfloat[Spectrogram reconstructed from the second layer]{\includegraphics[trim=2.9cm 0cm 0cm 0cm,clip=true,totalheight=0.65\columnwidth]{img/Bootleg/second_layer}\label{fig:LLFs:second}}
      \subfloat[Spectrogram reconstructed from the third layer]{\includegraphics[trim=2.9cm 0cm 0cm 0cm,clip=true,totalheight=0.65\columnwidth]{img/Bootleg/third_layer}\label{fig:LLFs:third}}            
      \caption{A song spectrogram reconstructed from different layers of a deep learning network.}
      \label{fig:LLFs:features}          
\end{figure}


Unlike \textit{hand-crafted} features, features learned using deep learning methods are not easily interpretable since there is not a clear mapping with specific acoustic cues. To show the input characteristics captured by the \textit{learned} features, it is possible to invert the process and reconstruct the input starting from the features. As an example, Figure \ref{fig:LLFs:features} shows a song spectrogram, and its reconstructed versions starting from \textit{learned} features at different abstraction layers. We notice that, at each layer, only characterizing frequencies of the spectrum are preserved, whereas less informative bands tend to be discarded.

Deep learning techniques perform non-linear transformations to input data in order to learn and extract salient information. Such techniques do not need to know in advance the target application of the extracted features, and therefore they extract a generic representation of the distribution of the input domain. For this reason, such techniques are named \textit{unsupervised} deep learning techniques, and the extracted features are referred to as \textit{unsupervised learned features}. 
%In particular, unsupervised deep learning techniques are able to learn and extract salient information from unlabeled data, i.e., from generic data of which they do not know any prior perform non-linear transformations to input data in order to learn and extract salient information in an unsupervised fashion. This technique is particularly suitable for feature learning process. %Although even a one-layer unsupervised learning algorithm could extract salient features, it has a limited capacity of abstraction. For this reason, a multi-layer approach can be used. Each layer is fed using the output of the lower-layer and each layer is designed to give a more abstract representation of the previous layer activations. In this way, higher-level abstractions that characterize the input could emerge.
In the following Sections, we provide a formalization for the unsupervised deep learning architecture we use in this work, the \textit{Deep Belief Network}, which is composed by stacking several layers of a 2-layer neural networks called \textit{Restricted Boltzmann Machine}. For the sake of completeness, in Section \ref{sec:LLFs:otherdeep} we discuss other deep learning architectures that  in the state of the art to address Music Information Retrieval tasks.


%
%\begin{figure}[t]
%%\captionsetup[subfigure]{justification=centering}
%%\centering
%    \subfloat[Input]{\includegraphics[width=.22\textwidth]{img/Bootleg/original_input} \hfil
%    \subfloat[First Layer]{\includegraphics[width=.22\textwidth]{img/Bootleg/first_layer} \hfil
%    \subfloat[Second Layer]{\includegraphics[width=.22\textwidth]{img/Bootleg/second_layer} \hfil
%    \subfloat[Third Layer]{\includegraphics[width=.22\textwidth]{img/Bootleg/third_layer} \hfil        
%	\caption{A song spectrogram (a), and its reconstructed versions starting from $learned$ features at different abstraction levels (b)(c)(d) .}
%	\label{fig:LLFs:features}      
%\end{figure}


\subsection{Restricted Boltzmann Machines}\label{sec:Bootleg:deep}
A Restricted Boltzmann Machine (RBM) is an \textit{energy-based} generative model, i.e., a model that associates a scalar energy to each configuration of the variables of interest, by assigning lower values to the samples that are likely to belong to that distribution. More formally, the energy-based models define a probability distribution through an energy function:
\begin{equation}
P(x)=\frac{e^{-\E(x)}}{Z},
\label{eq:probEnergy}
\end{equation}
where $Z$ is a normalization factor called \textit{partition function}, defined as 
\begin{equation}
Z=\sum_{\tilde{x}} e^{-\E(\tilde{x})},
\end{equation}
whose sum (or integral, if $x$ is continuous) spans the input space. The training of an energy-based model corresponds to shape the energy function such that low energy values are associated to plausible configurations, i.e., configuration of the input variable that are more likely to occur.

The energy function have different formulations. A common formulation is formalizes the energy function as a sum of components, where each component is called \textit{expert} $f_i$:
\begin{equation}
\E(x)=\sum_i f_i(x)
\end{equation}
\begin{equation}
\text{so that }P(x)\propto=\prod_i P_i(x) \propto \prod_i e^{f_i(x)}.
\label{eq:HLFs:experts}
\end{equation}
The probability distribution is therefore modeled as a \textit{product of experts}, where each expert $P_i(x)=e^{-f_i(x)}$ can be seen a detector of implausible configurations of $x$: if $x$ is implausible and then some $P_i(x)=0$, the total $P(x)=0$. With this formulation, each expert behaves as a constraint, which makes the Energy model a distributed representation.

\begin{figure}[tbp]
	\centering
	\includegraphics[width=0.6\columnwidth]{img/Bootleg/RBM_scheme_latexit.pdf}
	\caption{Restricted Boltzmann Machine topology.}
	\label{fig:Bootleg:RBMScheme}
	\vspace{-1em}
\end{figure}

The RBM is structured as a two-layer network of neurons: i) a visible layer $\mathbf{v} \in \mathbb{R}^{N}$ (the input layer); ii) a hidden layer $\mathbf{h} \in \mathbb{R}^{M}$ (see Fig.~\ref{fig:Bootleg:RBMScheme}), where $N$ and $M$ are the number of visible and hidden neurons respectively, which is decided in the architecture design step. Neurons in different layers are fully connected, whereas no connection exists between neurons of the same layer. The probability distribution is hence formalized as 
\begin{equation}
P(\mathbf{v},\mathbf{h})=\frac{e^{-\E(\mathbf{v},\mathbf{h})}}{Z},
\end{equation}
but, since only $\mathbf{v}$ is observed, we care about the marginal probability:
\begin{equation}
P(\mathbf{v})=\sum_\mathbf{h} \frac{e^{-\E(\mathbf{v},\mathbf{h})}}{Z},
\end{equation}
with a sum over all the hidden space (i.e., the hidden variables $\mathbf{h}$ that can occur jointly with $\mathbf{v}$).

We introduce the notation of \textit{free energy} to have a notation similar to the one of Eq. \ref{eq:probEnergy}:
\begin{equation}
P(\mathbf{v})=\frac{e^{-\FE(\mathbf{v})}}{\sum_{\tilde{\mathbf{v}}}e^{-\FE(\tilde{\mathbf{v}})}},
\end{equation}
\begin{equation}
\text{with }\FE(\mathbf{v})=-\log\sum_\mathbf{h} e^{-\text{Energy}(\mathbf{v},\mathbf{h})}
\label{eq:LLFs:freeenergy}
\end{equation}

In an RBM, the energy function for a configuration of input and hidden variable, is defined as \cite{Bengio2009}
\begin{equation}
\E(\mathbf{v},\mathbf{h}) = -\mathbf{b}^\top\mathbf{v} -\mathbf{c}^\top\mathbf{h}-\mathbf{h}^\top\mathbf{W}\mathbf{v},
\label{eq:HLFs:energyRBM}
\end{equation}
where $^\top$ indicates the transpose of a vector, while $\mathbf{b} \in \mathbb{R}^{N \times 1}$, $\mathbf{c} \in \mathbb{R}^{M\times 1}$, and $\mathbf{W} \in \mathbb{R}^{M \times N}$ are the parameters of the network. %To find the aforementioned parameters (i.e., $\mathbf{W}$, $\mathbf{b}$, and $\mathbf{c}$), 
In the RBM, the free energy function is defined as \cite{Bengio2009}
\begin{equation}
\FE(\mathbf{v}) = -\log \sum_{\mathbf{h} \in \mathcal{H}_\mathbf{v}}e^{-E(\mathbf{v},\mathbf{h})},
\end{equation}
where the sum is extended to the set $\mathcal{H}_\mathbf{v}$ of possible outputs $\mathbf{h}$, given an input $\mathbf{v}$ (i.e., we make some hypotheses on possible vectors $\mathbf{h}$ associated to a vector $\mathbf{v}$). 

The Free Energy then becomes: 
\begin{equation}
\FE(\mathbf{v})=-\mathbf{b}\T \mathbf{v} -\sum_i \log \sum_{h_i} e^{\mathbf{h}_i \mathbf{W}_i \mathbf{v}}
\end{equation}
and the conditional probability $P(\mathbf{h}|\mathbf{v})$ is now very easily tractable:
\begin{equation}
P(\mathbf{h}|\mathbf{v})=\prod_i P(\mathbf{h}_i|\mathbf{v}).
\end{equation}
The conditional distribution can be seen as a product of experts as in Eq. \ref{eq:HLFs:experts}, where each hidden neuron is an expert that partitions the input space. The conditional probability of the hidden variable $\mathbf{h}$ is simply a product of the conditional probabilities of the single units, i.e., we obtain the output for each of the hidden unit as  
\begin{equation}
P(\mathbf{h}_i|\mathbf{v})= \text{sigm}(\mathbf{c}_i + \mathbf{W}_i \mathbf{v}),
\end{equation}
where the $\text{sigm}(\mathbf{x})=(1+exp(-\mathbf{x}))^-1$ is the sigmoid function applied element-wise. Since the energy is symmetric (from Eq.\ref{eq:HLFs:energyRBM}), we can easily state that 
\begin{equation}
P(\mathbf{v}|\mathbf{h})=\prod_i P(\mathbf{v}_i|\mathbf{h})
\label{eq:LLFs:generatingx}
\end{equation} and $P(\mathbf{v}_i|\mathbf{h})= \text{sigm}(\mathbf{b}_i + \mathbf{W}\T_{.j} \mathbf{v})$, where $\mathbf{W}_{.j}$ is the $j$-th column of $\mathbf{W}$. 

%More formally, let us consider the vector $\mathbf{v} \in \mathbb{R}^{N\times 1}$ representing the data under analysis (e.g., samples of the spectrum of an audio excerpt), and the vector $\mathbf{h} \in \mathbb{R}^{M\times 1}$ representing the output of the RBM whose input is $\mathbf{v}$. Each neuron of the visible layer is a sample of $\mathbf{v}$, while each neuron of the hidden layer is a sample of $\mathbf{h}$. The goal of training an RBM is to find the parameters of the network that enable to compute $\mathbf{h}$ (i.e., our feature vector) from $\mathbf{v}$ (i.e., our data), so that $\mathbf{h}$ contains salient information that characterizes $\mathbf{v}$.
%With this in mind, and 
Given a set $\mathcal{V}$ of training data vectors $\mathbf{v}$, the network parameters are estimated as 
\begin{equation}
\{\hat{\mathbf{W}}, \hat{\mathbf{b}}, \hat{\mathbf{c}}\} = \argmin{\mathbf{W}, \mathbf{b}, \mathbf{c}} \prod_{\mathbf{v} \in \mathcal{V}} F(\mathbf{v}).
\label{eq:training}
\end{equation}
In other words, we search for the parameters of the network that better explain all the possible hypothesized outputs.

In order to find the network parameters to shape the Energy Function, different iterative algorithms can be applied. The most widely used, thanks to its relatively low computational cost is the Contrastive Divergence ($CD-k$) algorithm. A proper iterative algorithm would need a full Monte-Carlo sampling of the hidden space, which is not tractable. The $CD-k$ algorithm offers an easier solution by using an approximation in the computation of the Free Energy and exploiting the generative power of the RBM. 

The Free Energy is supposed to be estimated over all the possible hidden configuration given a certain input, as defined in \ref{eq:LLFs:freeenergy}. Instead, we substitute the sum over the space of the hidden configurations with a single sample, the one given by the parameters at the $j$-th iterative step  $\{\hat{\mathbf{W}}^{(j)}, \hat{\mathbf{b}}^{(j)}, \hat{\mathbf{c}}^{(j)} \}$. This will introduce variance in the approximation of the gradient, which is however reduced by constantly updating the parameters with the iterative procedure.

As far as the generative power of the RBM is regarded, from a configuration of the hidden variable $h$ we generate a new sample $x$ as defined in Equation \ref{eq:LLFs:generatingx}. We then perform a Gibbs Sampling by starting with a real visible input $v=x^{(0)}$, from which we generate $h^{(0)}$, that generates $x^{(1)}$, and so on:
%\begin{equation}
%\begin{array}{c}
\begin{align*}
x^{(0)} \sim \hat{P}(x)\\
h^{(0)} \sim P(h|x^{(0)})\\
x^{(1)} \sim P(x|h^{(0)})\\
...\\
x^{(k)} \sim P(x|h^{(k-1)}),
\end{align*}
%\end{array}
%\end{equation}
where  $\hat{P}$ is the empirical distribution, i.e., the distribution of the input, whereas $P$ is the model distribution.
The $CD-k$ algorithm perform $k$ sampling steps to generate a new sample. Since this sample has been generated by the model, it is less likely to belong to the empirical distribution. We want to shape the energy function in order to have low values for variables from the input data, which belong to the empirical distribution. The $CD-k$ algorith considers the generated data $x^{k>0}$ as negative samples, and it shapes the energy function to obtain higher values where generated samples are more highly distributed. While higher $k$ achieve better performance, $CD-k$ has proven to be effective even with $k=1$ \cite{Bengio2009}.

The vector of hidden neurons $\mathbf{h}$ are used as a representation of the input vector $\mathbf{v}$ containing the most salient information about it. However, RBM has only one representation layer, that cannot extract abstract information from the signal. The Deep Belief Network provides such depth of representation.

% ----- DBN -----
\subsection{Deep Belief Network}
A Deep Belief Network (DBN) \cite{Hinton2006} is a deep learning network architecture that is built as the stack of Restricted Boltzmann Machines (RBMs) on top of each other. Each layer $k$ takes as input the hidden vector $\mathbf{h}^{(k-1)}$ of the previous one (as in Fig.~\ref{fig:LLFs:DBN}) in order to produce the more abstract features $\mathbf{h}^{(k)}$. The optimal number of layers (i.e., its \textit{depth} $K$) and the number of nodes for each layer $H^{(k)}$ depends on the problem that the network is designed to address, and it is difficult to be defined a-priori. The more are the neurons and the deeper is the network, the better will be the performance but also the more expensive will be the training of the network.

DBNs have been effectively employed for several tasks, such as musical genre recognition \cite{Hamel2010}, music emotion recognition \cite{Schmidt2011}, and other audio classification problems \cite{Humphrey2013}. 
\begin{figure}[tbp]
  \centering
  \includegraphics[width=.6\textwidth]{img/MSA/deep.pdf}
  \caption{General representation of a DBN.}
    \label{fig:LLFs:DBN}
\end{figure} 

A DBN is trained by individually training the RBMs of which it is composed. First, the bottom layer is trained with the samples of the training set $\mathcal{V}$, obtaining the representation set
$$\mathcal{H}^{(1)}=\{P(\mathbf{h}|\mathbf{v}) \; \forall \mathbf{v} \in \mathcal{V}\}.$$ 
Then, $\mathcal{H}^{(1)}$ is fed to the RBM in the second layer in order to train it, generating the representation set $\mathcal{H}^{(2)}$, and so on for all the layers. 

This kind of architecture allows the RBMs in the top layers to combine the properties learned in the lower layers in order to provide a more abstract representation of the input data. Given a new data input $\mathbf{v}$, the output of the DBN is its multi-layer representation, i.e., the collection of all the hidden layers  $\mathbf{h}^{(k)}$ with $k=1,...K$, each one providing a different level of abstraction.

This training is made to learn a generic representation of the input data, hence unlabeled data is sufficient in this step. The learned parameters can be \textit{fine-tuned} in order to learn a target-oriented representation, i.e., a representation that is more useful for a given task. In our work we employ unsupervised features only

\subsection{Other deep architectures}\label{sec:LLFs:otherdeep}
In the latest years, several other deep architectures have been proposed and tested for the representation of the signal domain.

The autoencoders, or autoassociators \cite{Bengio2009}, are a class of neural networks that use a hidden layer to learn a representation of a visible layer such that the visible layer itself can be reconstructed from the hidden layer. The hidden units are computed as a non-linear transformation of a linear combination of the input units, which can be the sigmoid function or other non-linear functions. A popular non-linear transformation is the Rectified Linear Unit (ReLU) that is defined as $f(x)=\text{max}(0,x)$, which has the advantage of being piecewise linear, hence easily tractable for the computation of the gradient \cite{Zeiler2013}. Before training an autoencoder, some noise can be added to the input, in order to avoid overfitting issues and enforce the training of the network (\textit{denoising autoencoders} \cite{vincent2008extracting}). By stacking several layers of autoencoders together, with an approach similar of the DBNs, we obtain a deep learning architecture named \textit{stacked autoencoders}. Denoising stacked autoencoders have been employed in Music Information Retrieval by \cite{maillet2009steerable} to extract a music representation for the automatic generation of playlists.

DBNs and stacked autoencoders usually work on a frame-level representation of the audio input, attempting at capturing the salient features. This approach does not take the time into consideration. %of frequency dimension 
%, since it treats all the frequency bins and all the time frames in the same manner. Two are the main architectures that address this issue. The former are the Recurrent Neural Networks, that provide the feature extraction with memory, in order to store salient information from the past samples. The latter are the Convolutional Neural Networks, that extract the feature representation from several consecutive frames and make them invariant with respect to the time and the frequency domain. 
Two are the main deep architectures that include time information in the model: the Recurrent and the Convolutional Neural Networks.

The Recurrent Neural Networks (RNNs) are able to learn a representation of the input that takes the time dimension into account by keeping memory of the past samples. In order to do so, they take as input both the data of the current frame and some information on the previous ones. More formally, given an instant $t$, which is represented by a time-frame input $x_t$, the network returns the hidden representation of the frame $h_t$ as:
\begin{equation}
h_t=f(x_t, h_{t-1}),
\end{equation}
where $f$ is the generic function for the generation of the output (usually a non-linear function of a linear combination of the input). The presence of $h_{t-1}$ as input is important because it is itself generated as $f(x_{t-1}, h_{t-2})$, i.e., it encodes information from the previous samples \cite{mesnil2013investigation}. The RNNs are usually learned through a supervised training step to learn a representation oriented for a task. The RNNs have quickly achieved and sometimes outperformed the state-of-the-art performance in several MIR tasks, such as beat tracking \cite{bock2014multi}, music transcription \cite{sigtia2016end} or emotion recognition \cite{Weninger2014}.

The Convolutional Neural Networks (CNN) are a popular architecture that have been originally designed for image recognition and have been later used in music information \cite{lee2009convolutional}. The basic idea is to use a filter, named \textit{kernel}, which is convolved with the input in different locations, to collect all the outcome of such filtering and use the pooling of the outcomes as the hidden layer representation. The training step concerns the learning of the weights of the kernel, to obtain a translation-invariant hidden layer representation, due to the pooling of the filtering outcomes from different location. As an example, let us suppose that the input is a set of frames from a spectrogram, i.e., a 2D signal: depending on the direction of the shifting of the kernel, it is possible to make a CNN either time-invariant or frequency-invariant. This addresses, for example, the problem of detecting whether a certain acoustic event occur in a spectrogram: once the kernel is learned, the CNN will allow to capture the event in whichever moment it occurs. With the same approach, we detect frequency-shifted acoustic events. The CNNs are in fact highly flexible for the audio analysis whenever the invariancy to time or frequency shifting is desirable. For this reason, CNNs have been employed for structural analysis \cite{ullrich2014boundary} as well as music classification \cite{dieleman2011audio}, annotation \cite{dieleman2014end} and recommendation \cite{van2013deep}.

Due to the tremendous impact of the deep learning on the state of the art of machine learning, many frameworks have been proposed for the design and implementation of the architectures. In this work, we use the \textit{Theano} Python library \cite{Bergstra2010}, which provides automatic gradient computation by composing and solving a computational graph derived from the architecture. Other frameworks include Caffe \cite{jia2014caffe} and TensorFlow \cite{tensorflow}.

\section{Final considerations}
The formalization of the signal domain involves the physical world of the sound waves, the perceptual world of the listening and the musicological aspects of the content. Each of this world is composed by several hierarchical layers of complexity and abstraction.

In order to extract a generic representation of the audio signal, or its musical content, we can extract LLFs or MLFs respectively. They are reliable descriptors for the signal domain, which have been manually designed by researchers to capture a specific aspect of the signal. In order to extract them, it is necessary to understand which characteristics matter for the addressing of the problem and which features can capture such characteristics. We conducted a research activity in the context of the Networked Music Performance to develop a better and deeper understanding of the semantics carried by LLFs and MLFs. This study also highlighted how these features can be effectively used to define the signal domain of a real problem and develop novel solutions.

However, sometimes the needed characteristics are not known in advance and the features who are supposed to capture them do not ensure the required precision. We address these issues by using the deep learning techniques to provide a generic representation of the audio signal with several layers of abstraction. Since we can hardly infer an interpretation of the learned features \cite{choi2015auralisation}, the design effort is shifted from features to architectures. Several architectures have been proposed in the literature and others can be created with ad-hoc solution that take into account the nature of music. Nevertheless, it was proven that deep learning techniques are effective for extracting a feature representation to address problems related to the signal domain. 
